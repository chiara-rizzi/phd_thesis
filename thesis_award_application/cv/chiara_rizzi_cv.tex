
\documentclass[11pt,a4paper]{moderncv}

% moderncv themes
%\moderncvtheme[blue]{casual} oldstyle banking classic           
\moderncvtheme[blue]{classic}  
\usepackage[utf8]{inputenc}                   % replace by the encoding you are using

% adjust the page margins
\usepackage[scale=0.8, bottom=2.7cm, top=2.6cm, left=2.5cm, right=2.5cm]{geometry}
%\setlength{\hintscolumnwidth}{3cm}						% if you want to change the width of the column with the dates
%\AtBeginDocument{\setlength{\maketitlenamewidth}{6cm}}  % only for the classic theme, if you want to change the width of your name placeholder (to leave more space for your address details
%\AtBeginDocument{\recomputelengths}                     % required when changes are made to page layout lengths

% personal data
\firstname{Chiara}
\familyname{Rizzi}
%\title{Curriculum Vitae}               % optional, remove the line if not wanted
% \address{street and number}{postcode city}    % optional, remove the line if not wanted
%\mobile{+393409413464}                    % optional, remove the line if not wanted
\phone{+393409413464}                      % optional, remove the line if not wanted
% \fax{fax (optional)}                          % optional, remove the line if not wanted
\email{chiara.rizzi@cern.ch}                      % optional, remove the line if not wanted               % optional, remove the line if not wanted
% \extrainfo{additional information (optional)} % optional, remove the line if not wanted
%  \photo[64pt]{picture}                         % '64pt' is the height the picture must be resized to and 'picture' is the name of the picture file; optional, remove the line if not wanted
% \quote{Some quote (optional)}                 % optional, remove the line if not wanted

%\fancyfoot[c]{Chiara Rizzi, chiara.rizzi@cern.ch}

% to show numerical labels in the bibliography; only useful if you make citations in your resume
% \makeatletter
% \renewcommand*{\bibliographyitemlabel}{\@biblabel{\arabic{enumiv}}}
% \makeatother

% bibliography with mutiple entries
%\usepackage{multibib}
%\newcites{book,misc}{{Books},{Others}}

%\nopagenumbers{}                             % uncomment to suppress automatic page numbering for CVs longer than one page
%----------------------------------------------------------------------------------
%            content
%----------------------------------------------------------------------------------

\newcommand{\dayth}{$^\mathrm{th}\;$}

\begin{document}

%\begin{minipage}{0.49\textwidth}
\maketitle
%\end{minipage}
%\begin{minipage}{0.49\textwidth}
%\section{Contacts}
%\cvline{Born}{July 27th 1989}
%\cvline{Citizenship}{Italian}
%\cvline{}{chiara.rizzi$@$cern.ch}
%\cvline{Address}{9bis rue de Lyon, 01630 Saint Genis Pouilly (France)}
%\cvline{}{+39 3409413464}
%\includegraphics[width=0.7\columnwidth]{./chiara_rizzi2.JPG}
%\par\bigskip
%\par\bigskip
%\par\medskip
%\par\smallskip
%\par\smallskip
%\end{minipage}

%\par\bigskip
%\par\bigskip
%\par\bigskip
%\par\bigskip

%\section{Profile}
%PhD student in particle physics, in the last year of the studies, with several years of experience in international laboratories (CERN and Fermilab). 
%\par\bigskip
%\begin{itemize}
%\item Solid background in statistics and data analysis.
%\item Experienced in several programming languages and statistical software (R, ROOT, Python libraries). 
%\item Strong communication skills and experience of teamwork in multi-cultural international laboratories.
%\end{itemize}
%\cvitem{}{\listitemsymbol Bla}
%\cvitem{}{\listitemsymbol Excellent communication skills and experience of teamwork in multi-cultural environments}

\section{Current Position}
\cventry{Nov 2018 - present}{CERN Fellowship}{}{}{}{} 


\section{Education}
%\cventry{Currently}{CERN Fellowship}{}{}{}{}
\cventry{2018}{PhD in Physics}{Universitat Autònoma de Barcelona}{La Caixa-Severo Ochoa International Doctoral Fellowship}{}{Thesis: ``Searches for supersymmetric particles in final states with multiple top and bottom quarks with the ATLAS detector'', supervisor: Prof. Aurelio Juste Rozas.}{}
\cventry{2014}{Master in Physics}{Universit\`a degli Studi di Milano}{}{\textit{110/110} summa cum laude}{Thesis: ``Search for scalar top in final states with missing transverse momentum and two tau leptons in 8 TeV p-p collisions collected by the ATLAS detector'', supervisor: Dr. Tommaso Lari.}{}
\cventry{2011}{Bachelor of Science in Physics}{Universit\`a degli Studi di Milano}{}{\textit{110/110} summa cum laude}{Thesis: ``Reconstruction of top quark pairs with the ATLAS Experiment at the LHC'', supervisor: Dr. Tommaso Lari.}
\cventry{2008}{High School Diploma}{Liceo Classico Alessandro Volta}{Como}{}{}

%\section{University Courses and Scores}
%\subsection{Bachelor}
% 
%\cvline{Classes} {During university I've attended all the obligatory classes and, as elective ones: 
%\begin{itemize}
%\item General Relativity
%\item Partial Differential Equations
%\item Computational Methods of Physics
%\end{itemize}}
%\cvline{Grades}{Average grade: 28.01198/30}


%\subsection{Master}
%\cvline{Grades}{Average grade: 29.4/30}


\section{Additional Trainings}
\cventry{2017}{CERN-Fermilab HCPSS}{Hadron Collider Physics Summer School}{}{Geneva}{}
\cventry{2016}{CERN-JINR ESHEP}{European School of High Energy Physics}{}{Skeikampen}{}{}{}
\cventry{2015}{MCnet School}{Summer School on Monte Carlo Event Generators}{}{Louvain}{}
\cventry{2015}{Terascale Statistics School}{}{DESY, Hamburg}{}{}{}
\cventry{2013}{Fermilab Summer Intern}{Summer Student Program}{Chicago}{}{}  
\cventry{2012}{HASCO}{Hadron Collider Summer School}{}{Göttingen}{}
%\cventry{2006}{Exchange student}{Lincoln East High School}{Lincoln (Nebraska)}{}{}%{During high school I spent one year in the United States as 
%\cventry{2017}{CERN-Fermilab Hadron Collider Physics Summer School}{CERN, Geneva}{}{}{}
%\cventry{2016}{CERN-JINR European School of High Energy Physics}{}{}{}{}{}
%\cventry{2015}{MCnet School}{30 August - 4 September}{Louvain}{}{}{}
%\cventry{2015}{Terascale Statistics School 2015}{23-27 March}{DESY, Hamburg}{}{}{}
%\cventry{2013}{Fermilab Summer Intern}{Summer Student Program, August-September}{}{}{}  
%\cventry{2012}{HASCO}{Hadron Collider Summer School}{15-27 July}{Georg-August-Universtät, Göttingen}{}
%\cventry{2006}{Exchange student}{Lincoln East High School}{Lincoln (Nebraska)}{}{}%{During high school I spent one year in the United States as an exchange student, living with an American host family.}
%\cventry{2015}{ATLAS Shift Training}{3-4 June}{CERN, Geneva}{}{}{}
%\cventry{2015}{HistFitter Tutorial}{30-31 March}{DESY, Hamburg}{}{}{}
%\cventry{2015}{Flavour Tagging Workshop}{11-13 February}{CERN, Geneva}{}{}
%\cventry{2014}{ATLAS Offline Software Tutorial}{6-10 October}{CERN, Geneva}{}{}{}
%\cventry{2014}{Third Generation Squark Workshop}{23-27 June}{Milan}{}{}

\section{Research Experience}
During my bachelor, master and PhD studies I had the opportunity to carry out my research in the ATLAS Collaboration and, in the context of a summer student program, I also worked in the CDF Experiment. 
For my bachelor thesis and at CDF I worked on analyses in the framework of top quark physics, while during my master thesis and PhD the main focus of my research has been on R-parity-conserving (RPC) Supersymmetry (SUSY) searches. SUSY, one of the most promising extensions of the Standard Model (SM), predicts the existence of superpartners for all the SM particles; in the RPC minimal supersymmetric standard model SUSY particles are produced in pairs and the lightest supersymmetric particle (LSP) is stable. In parallel during the PhD I have also contributed to the TileCal luminosity group, to the $b$-tagging calibration effort, and to the development of a tool designed to overcome the low statistics of Monte Carlo (MC) simulations in events with high $b$-jets multiplicity. Below I detail the main activities where I have been involved and my contributions, starting from the most recent ones.

\par\medskip

\cvline{\textbf{TileCal} 2017-2018}{The ATLAS hadronic tile calorimeter (TileCal) is one of the ATLAS sub-systems sensitive to the value of instantaneous luminosity: a higher luminosity corresponds to a higher particle flux, which in turn leads to a higher current drawn by the photomultiplier tubes (PMT). I studied how the PMT response depends on the peak luminosity, and how this impacts the TileCal luminosity determination.}

\cvline{\textbf{Higgsinos} 2016-present}{In natural SUSY models, where SUSY solves the fine-tuning in the Higgs mass determination, the SUSY particles that give relevant corrections to the Higgs mass are required to be light. These are the partners of the Higgs boson (higgsinos), the top (stops) and the gluons (gluinos). I have performed a search targeting higgsino pair production, where each higgsino decays into Higgs boson and gravitino, the SUSY partner of the graviton, that escapes undetected. The Higgs boson decays into $b\overline{b}$ pair with 58\% branching ratio, so a search for final states with four $b$-jets and missing transverse momentum ($E_\mathrm{T}^{\mathrm{miss}}$) is very sensitive to this scenario. The dominant background is top quark pair production in association with heavy flavour quarks. In this search, where I have been analysis contact, I have worked on the channel targeting signals with high masses; I have defined the analysis strategy, optimized the analysis regions to maximize the sensitivity, verified the background modelling and performed the statistical analysis.
%During my PhD I have actively contributed to two SUSY searches, targeting two different natural SUSY models, both leading to a final state with multiple b-jets and missing transverse momentum (MET). \newline 
}

\cvline{\textbf{Gluinos} 2015-present}{
A major activity that I have carried out since the beginning of my PhD has been a search for gluino pair production, with a decay chain involving four top or bottom quarks and the LSP, leading to a final state with four $b$-jets, $E_\mathrm{T}^{\mathrm{miss}}$ and a variable number of leptons depending on the top quarks decay. Also in this search, top quark pair production in association with heavy flavour jets constitutes the dominant background. I have been strongly involved in all the iterations of this analysis, that during the LHC Run2 has produced two papers and several conference notes. In the different iterations I have worked on various aspects of the analysis, from analysis optimization and extensive studies of the modelling of the background, to the final statistical analysis and limit setting, and I am now analysis contact.}

\cvline{\textbf{$b$-tagging} 2014-2017}{Low MC statistics is often a limitation in analyses selecting events with high $b$-jets multiplicity. This problem can be overcome if, instead of selecting only the MC events with many $b$-jets, all the events are kept and weighted by the probability that each event contains a certain number of $b$-tagged jets (truth tagging). I have developed and validated a tool to easily apply this method, which is now used in several analyses.\newline
I have also joined the $b$-tagging calibration group, by contributing to the effort toward a data-based calibration of the $b$-tagging algorithms for high-p$_\mathrm{T}$ jets. This study aims at using the distribution of the impact parameter significance of the tracks inside the jets to measure in data the different flavour fractions before and after applying the $b$-tagging algorithm, allowing therefore to measure the algorithm efficiency on data and compare it with the efficiency from MC simulation. }

\cvline{\textbf{Stop, stau} 2013-2014}{During my master thesis I have analyzed the data collected in 2012 by the ATLAS Experiment, selecting events with one hadronically and one leptonically decaying tau candidates, $b$-jets and missing momentum to search for direct  production of stop pairs, decaying via a scalar tau (stau) to a gravitino. I have carried out most of the steps of the analysis, focusing in particular on the analysis optimization and on the modelling of the major backgrounds: top quark pairs and W bosons in association with jets.}

\cvline{\textbf{Top Di-Tau}\newline 2013}{As a summer student I have contributed to a CDF analysis measuring the top pair production cross section and the top decay branching fraction into tau lepton, tau neutrino and bottom quark, and placing limits on the branching fraction of the top quark into charged Higgs. When I joined the analysis team the analysis was already structured, and my contribution has been revising and performing the statistical interpretation and producing the material for the paper.}

\section{Responsibility Roles in ATLAS}
\cvline{2018-present}{Analysis contact for the ATLAS SUSY strong-production multi-b analysis.}
\cvline{2017-present}{Analysis contact for the ATLAS SUSY electroweak multi-b analysis.}


\section{Publications}
\color{color1}\begin{large}ATLAS Publications\end{large}\\ \color{black}
As a member of the ATLAS Collaboration I am author of all the ATLAS publications since 2015. Below I highlight the ones where I have given a significant personal contribution.
\begin{itemize} 
\item  ATLAS Collaboration, ``Search for pair production of higgsinos in final states with at least three $b$-tagged jets in $\sqrt{s} = 13$ TeV pp collisions using the ATLAS detector'',  In: Phys.Rev. D98 (2018) no.9, 092002. 
arXiv:\href{https://arxiv.org/abs/1806.04030}{\color{color1}1806.04030 [hep-ex]}.
\item  ATLAS Collaboration, ``Search for Supersymmetry in final states with missing transverse momentum and multiple $b$-jets in proton–proton collisions at $\sqrt{s} = 13$ TeV with the ATLAS detector'', JHEP 06 (2018) 107. arXiv:\href{https://arxiv.org/abs/1711.01901}{\color{color1}1711.01901 [hep-ex]}.
\item ATLAS Collaboration, ``Search for pair production of gluinos decaying via stop and sbottom in events with $b$-jets and large missing transverse momentum in pp collisions at $\sqrt{s} = 13$ TeV with the ATLAS detector''. In: Phys. Rev. D94.3 (2016), p. 032003. arXiv:\href{https://arxiv.org/abs/1605.09318}{\color{color1}1605.09318 [hep-ex]}.
\item ATLAS Collaboration, ``Search for direct top squark pair production in final states with two tau leptons in pp collisions at $\sqrt{s} = 8$ TeV with the ATLAS detector''. In: Eur. Phys. J. C76.2 (2016), p. 81. DOI: 10.1140/epjc/s10052-016-3897-z. arXiv:\href{https://arxiv.org/abs/1509.04976}{\color{color1}1509.04976 [hep-ex]}.
\end{itemize}
\par\medskip
\color{color1}\begin{large}Other Publications\end{large}\color{black}
\begin{itemize} 
\item CDF Collaboration. ``Study of Top-Quark Production and Decays involving a Tau Lepton at CDF and Limits on a Charged-Higgs Boson Contribution''. In: Phys. Rev. D89.9 (2014), p. 091101. arXiv:\href{https://arxiv.org/abs/1402.6728}{\color{color1}1402.6728 [hep-ex]}.
\end{itemize}

\section{Conference Notes}
\begin{itemize}
\item  ATLAS Collaboration, ``Search for supersymmetry in final states with missing transverse momentum and multiple $b$-jets in proton-proton collisions at $\sqrt{s} = 13$ TeV with the ATLAS detector'', \href{https://cds.cern.ch/record/2632347}{\color{color1}ATLAS-CONF-2018-041}.
\item  ATLAS Collaboration, ``Search for pair production of gluinos decaying via top or bottom squarks in events with $b$-jets and large missing transverse momentum in
$pp$ collisions at $\sqrt{s} = 13$ TeV with the ATLAS detector'', \href{https://atlas.web.cern.ch/Atlas/GROUPS/PHYSICS/CONFNOTES/ATLAS-CONF-2016-052/}{\color{color1}ATLAS-CONF-2016-052}.
\end{itemize}

\section{Internal Notes}
\begin{itemize}
\item  Aine Kobayashi, Anna Shcherbakova, Chiara Rizzi, Aurelio Juste, Sara Strandberg, Yuji Enari, ``Measurement of the $b$-tagging efficiency of the MV1 algorithm for High p$_T$ in $pp$ collisions at $\sqrt{s} = 8$ TeV using multi-jet events'', \href{https://cds.cern.ch/record/2032163}{\color{color1}ATLAS-CONF-2018-041}
\end{itemize}

\section{Conferences and Workshops}
\begin{itemize}
\item Talk: ``Multibin and shape fits in SUSY analyses'', ATLAS SUSY and Exotics Joint Workshop, 8-12 May 2017, Bucharest.
\item Talk: ``Search for gluino-mediated stop and sbottom pair production in events with $b$-jets and large missing transverse momentum'', ALPS2017, 15-20 April 2017, Obergurgl.
\item Poster: ``Search for pair production of gluinos in final states with many $b$-jets and $E_T^{\mathrm{miss}}$ at $\sqrt{s}=13$ TeV with the ATLAS detector'', LHCC Poster Session, 2 March 2016, CERN, Geneva.
\item Talk: ``Ricerche di un partner scalare del quark top in stati finali con due leptoni tau con l'esperimento ATLAS'', 100$^{th}$ Congress of the “Societ`a Italiana di Fisica'' (SIF), 22-26 September 2014, Pisa.
\end{itemize}

\iffalse
\section{ATLAS Internal Talks}

\cvline{Full Analysis Review}{For the high-mass channel: ``Search for pair production of higgsinos in final states with at least 3 $b$-tagged jets using the ATLAS detector in $\sqrt{s} = 13$ TeV $pp$ collisions'', July 28$^\mathrm{th}$ 2017, \href{https://indico.cern.ch/event/656418/contributions/2674288/attachments/1500826/2337270/FAR_hh4b_v2.pdf}{\color{color1}link}}

\cvline{TileCal Week}{``PMT non-linearity studies and implications in the TileCal luminosity calibration'', February 16th 2018, \href{https://indico.cern.ch/event/703027/contributions/2894678/attachments/1601891/2539936/chiara_pmt.pdf}{\color{color1}link}}


\cvline{ATLAS Open Presentation}{``Search for Supersymmetry in final states with missing transverse momentum and multiple $b$-jets in proton--proton collisions at $\sqrt{s} = 13$ TeV with the ATLAS detector'', March 16th 2017, \href{https://indico.cern.ch/event/622781/contributions/2512064/attachments/1429263/2194530/open_presentation.pdf}{\color{color1}link}  }


\cvline{Full Analysis Review}{For for ``Search for pair production of gluinos decaying via top or bottom squarks in events with $b$-jets and large missing transverse momentum in $pp$ collisions at $\sqrt{s}=13$ TeV with the ATLAS detector'', \href{https://indico.cern.ch/event/532982/contributions/2188946/attachments/1282936/1906831/FAR_Multib_chiara.pdf}{\color{color1}link} }

\cvline{ATLAS Open Presentation}{``Search for pair production of gluinos decaying via stop and sbottom in events with $b$-jets and large missing transverse momentum in $pp$ collisions at $\sqrt{s} = 13$ TeV with the ATLAS detector'', February 16ht 2016, \href{https://indico.cern.ch/event/496698/contributions/1175271/attachments/1229520/1801685/open_presentation.pdf}{\color{color1}link}  }

\cvline{B-tagging Plenary}{Talk at Flavour Tagging P\&P Meeting ``High-p$_\mathrm{T}$ $b$-tagging in di-jets events'', \href{https://indico.cern.ch/event/437597/contributions/1924966/attachments/1150999/1652290/Sept8.pdf}{\color{color1}link}}

\cvline{Full Analysis Review}{September 18th 2014, Full analysis review of the lepton-hadron channel for the analysis: ``Search for direct top squark pair production in final states with two tau leptons in pp collisions at $\sqrt{s}=8$  TeV with the ATLAS detector'', \href{https://indico.cern.ch/event/278649/contributions/1624956/attachments/509623/703359/FAR_Sept18.pdf}{\color{color1}link}}
\fi

\section{ATLAS Internal Talks}
\begin{itemize}
\item Talk at TileCal Week: ``PMT non-linearity studies and implications in the TileCal luminosity calibration'', February 16\dayth 2018, \href{https://indico.cern.ch/event/703027/contributions/2894678/attachments/1601891/2539936/chiara_pmt.pdf}{\color{color1}link}.
\item Full Analysis Review of the high-mass channel of: ``Search for pair production of higgsinos in final states with at least 3 $b$-tagged jets using the ATLAS detector in $\sqrt{s} = 13$ TeV $pp$ collisions'', July 28\dayth 2017, \href{https://indico.cern.ch/event/656418/contributions/2674288/attachments/1500826/2337270/FAR_hh4b_v2.pdf}{\color{color1}link}.
\item ATLAS Open Presentation: ``Search for Supersymmetry in final states with missing transverse momentum and multiple $b$-jets in proton--proton collisions at $\sqrt{s} = 13$ TeV with the ATLAS detector'', March 16\dayth 2017, \href{https://indico.cern.ch/event/622781/contributions/2512064/attachments/1429263/2194530/open_presentation.pdf}{\color{color1}link}.
\item Full Analysis Review of ``Search for pair production of gluinos decaying via top or bottom squarks in events with $b$-jets and large missing transverse momentum in $pp$ collisions at $\sqrt{s}=13$ TeV with the ATLAS detector'', June 1$^{\mathrm{st}}$ 2016, \href{https://indico.cern.ch/event/532982/contributions/2188946/attachments/1282936/1906831/FAR_Multib_chiara.pdf}{\color{color1}link}.
\item ATLAS Open Presentation: ``Search for pair production of gluinos decaying via stop and sbottom in events with $b$-jets and large missing transverse momentum in $pp$ collisions at $\sqrt{s} = 13$ TeV with the ATLAS detector'', February 16\dayth 2016, \href{https://indico.cern.ch/event/496698/contributions/1175271/attachments/1229520/1801685/open_presentation.pdf}{\color{color1}link}.
\item Talk at Flavour Tagging P\&P Meeting ``High-p$_\mathrm{T}$ $b$-tagging in di-jets events'', \href{https://indico.cern.ch/event/437597/contributions/1924966/attachments/1150999/1652290/Sept8.pdf}{\color{color1}link}.
\item Full analysis review of the lepton-hadron channel of: ``Search for direct top squark pair production in final states with two tau leptons in pp collisions at $\sqrt{s}=8$  TeV with the ATLAS detector'', September 18\dayth 2014, \href{https://indico.cern.ch/event/278649/contributions/1624956/attachments/509623/703359/FAR_Sept18.pdf}{\color{color1}link}.
\end{itemize}


\section{Teaching and Outreach}
\cvline {2017-present}{Supervision of master student (J. Urtasun Elizari) working at optimization of Supersymmetry searches. }
\cvline {2017}{Supervision of summer student (J. Aguado Lopez) working on Higgs boson reconstruction techniques, \href{https://cds.cern.ch/record/2295723/files/documentation_Jesus_Aguado.pdf}{\color{color1}link to his final report}.}
\cvline {2017}{Moderator for International Masterclasses.}
\cvline {2015-present} {CERN S'Cool LAB tutor.}
\cvline {2015-present} {Official CERN guide.}
\cvline {2012-2013} {University assistant for the Physics and Statistics Laboratory.} 
\cvline {2012-2013} {Tutor for high school and university students in physics and mathematics.}

\section{Technical Skills}
\cvline{Statistical Software} {Expert in ROOT, intermediate level in R, Wolfram Mathematica and several Python libraries (SciPy, pandas, NumPy). } %, \textbf{Statsmodels}
\cvline{Programming Languages} {Experienced in programming with C, C++, and Python, good knowledge of Bash.}
%\cvline{Markup Languages}{}
%\cvline{Version Control}{Git, svn}
%\cvline{Operating System} {Worked with: Linux (Ubuntu, Fedora, Debian), macOS (Mac OS X Lion, macOS Sierra), Windows}
\cvline{Other}{Experienced in using version control systems (Git, svn) and markup languages (LaTex, Markdown).}

\section{Languages}
\cvline{Italian}{Native language.}
\cvline{English}{Full professional proficiency.}
\cvline{French}{Beginner.}

\end{document}


%% end of file `template_en.tex'.



% \begin{itemize}%
% \item Quantum Mechanics I, II
% \item Introduction to Nuclear and Particle Physics
% \item Structure of Matter
% \item Computer science
% \item Numerical Analysis
% \item Mathematical Analysis I, II, II
% \end{itemize}}
% \cvline{}{ \begin{itemize}
% \item Geometry I
% \item Analytic Mechanics
% \item Electromagnetism
% \item Mechanics
% \item Thermodynamic
% \item Waves and Oscillations
% \item Chemistry 
% \item Laboratory 
%  \end{itemize}}

