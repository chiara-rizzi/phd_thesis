\documentclass[11pt,a4paper]{article}



\usepackage[utf8]{inputenc}
\usepackage[english]{babel}
\usepackage{amsmath}
\usepackage{amsfonts}
\usepackage{amssymb}
\usepackage{xspace}
\usepackage{graphicx}
\usepackage[left=2cm,right=2cm,top=1.85cm,bottom=1.85cm]{geometry}

%\usepackage[backend=bibtex,style=numeric,natbib=true,sorting=none]{biblatex}
%\addbibresource{two_page_summary.bib}
\usepackage{cite}
\usepackage[autostyle=true]{csquotes}
\bibliographystyle{atlasBibStyleWithTitle}

\title{Searches for supersymmetric particles in final states with multiple top and bottom quarks with the ATLAS detector}
\author{Chiara Rizzi}
\date{}

\newcommand{\cmsette}{\ensuremath{\sqrt{s}  =  7}\xspace}
\newcommand{\cmotto}{\ensuremath{\sqrt{s}  =  8}\xspace}
\newcommand{\cmtre}{\ensuremath{\sqrt{s}  =  13}\xspace}
\newcommand{\cmfour}{\ensuremath{\sqrt{s}  =  14}\xspace}
\newcommand{\cmpart}{\ensuremath{\sqrt{\hat{s}}}\xspace}
\newcommand{\lumi}{\ensuremath{\int L \, dt}}
\def\ipb{\mbox{pb$^{-1}$}} %  Inverse picobarns.
\def\ifb{\mbox{fb$^{-1}$}} %  Inverse femtobarns.
\def\inb{\mbox{nb$^{-1}$}}
\newcommand{\pp}        {\ensuremath{pp}\xspace}
\newcommand{\met}{\ensuremath{E_{\rm T}^{\mathrm{miss}}}\xspace}
\newcommand{\pt}{\ensuremath{p_{\rm T}}\xspace}
\newcommand{\ttbar}{\ensuremath{t\overline{t}}\xspace}
\newcommand*{\gravino}{\ensuremath{\tilde{G}}\xspace}
\renewcommand*{\stop}{\ensuremath{\tilde{t}}\xspace}
\newcommand*{\sbottom}{\ensuremath{\tilde{b}}\xspace}
\newcommand*{\gluino}{\ensuremath{\tilde{g}}\xspace}
\def\stilde{\widetilde}
\def\hino{\ensuremath{\tilde{H}}\xspace}
\def\mhino{\ensuremath{m(\tilde{H})}\xspace}
\newcommand*{\ninoone}{\ensuremath{\mathchoice%
      {\displaystyle\raise.4ex\hbox{$\displaystyle\tilde\chi^0_1$}}%
         {\textstyle\raise.4ex\hbox{$\textstyle\tilde\chi^0_1$}}%
       {\scriptstyle\raise.3ex\hbox{$\scriptstyle\tilde\chi^0_1$}}%
 {\scriptscriptstyle\raise.3ex\hbox{$\scriptscriptstyle\tilde\chi^0_1$}}}\xspace}
\newcommand*{\chinoonepm}{\ensuremath{\mathchoice%
      {\displaystyle\raise.4ex\hbox{$\displaystyle\tilde\chi^\pm_1$}}%
         {\textstyle\raise.4ex\hbox{$\textstyle\tilde\chi^\pm_1$}}%
       {\scriptstyle\raise.3ex\hbox{$\scriptstyle\tilde\chi^\pm_1$}}%
 {\scriptscriptstyle\raise.3ex\hbox{$\scriptscriptstyle\tilde\chi^\pm_1$}}}\xspace}


\begin{document}

\maketitle

At the LHC $pp$ collisions are used to probe the nature of particles in energy regimes that were not accessible before. 
After the discovery of the Higgs boson in 2012, the LHC is now continuing its operations with the dual goal 
of measuring the Higgs boson properties in great detail but also to continue the quest for new particles.
ATLAS is one of the two general purpose experiments on the LHC ring, and this thesis focuses on the data that is has collected between 
2015 and 2016, 
%in the first part of the LHC Run 2, 
at a center-of-mass energy \cmtre TeV, for a total  
integrated luminosity of approximately 36 \ifb. 

In this dissertation I discuss two searches for SUSY in final states enriched in $b$-jets, the collimated sprays of 
particles originating from the hadronization of a $b$-quark. 
I have been strongly involved in both analyses, 
and in this thesis I describe with more emphasis the topics where I have given direct contribution.

The Standard Model (SM) is the theory that as of today best describes the experimental results on subatomic particles. 
Nevertheless, there are strong theoretical and experimental arguments to believe that the SM is the low-energy limit 
of a more general theory, yet to be determined. 
Supersymmetry (SUSY) is one of the most promising SM extensions, 
addressing some of its shortcomings. SUSY predicts the existence of partners for  
SM particles, which cancel the SM corrections to the Higgs boson mass, solving the naturalness problem. 
In ``natural'' SUSY models, several particles are expected to be light and therefore observable at the LHC.
First of all higgsinos, which share the same tree-level mass parameter as the Higgs boson. 
Then top squarks (stops), that provide a one-loop correction to its mass, and gluinos, that give a two-loop correction since they 
contribute at one-loop at the stop mass. 
The requirement on the stop mass reflects also on the sbottom mass, to which is related through the weak-isospin symmetry. 
These particles are exactly the target of the searches presented in this thesis.
Furthermore, in the framework of R-parity conserving SUSY, supersymmetric particles are produced in pairs and the 
lightest supersymmetric particle (LSP) is stable. 
In several models the LSP is neutral and weakly interacting, providing a good candidate for dark matter.
 
The first analysis discussed in the thesis is a search for gluino pair production, where each gluino decays through a stop or a sbottom 
to respectively four top or four bottom quarks and the LSP, which is assumed to be neutral and stable.
Since the top quark decays to a $b$-quark and a $W$-boson, both the gluino decay chain through stop and the one through sbottom lead to a 
final state with multiple $b$-jets and missing transverse momentum (\met). 
This analysis employs two different strategies: cut-and-count, with several non-orthogonal SRs optimized to 
maximize the discovery significance for selected benchmark models, and multi-bin, with orthogonal SRs 
that are statistically combined in the maximum-likelihood fit 
to maximize the exclusion power of the analysis. 
In all the SRs semi-leptonic \ttbar+jets constitutes the dominant background.
No significant excess is found in any of the analysis regions. 
The largest deviation between expected and observed number of events 
is in SR-0L-HH, one of the multi-bin SRs, and it has a significance of approximately 
2.3 standard deviations. 
Exclusion limits in the m(\gluino)-m(\ninoone) are set for the two simplified models assuming 
100\% branching ratio (BR) for the gluino into $t\bar{t}\ninoone$ and $b\bar{b}\ninoone$, denoted as ``Gtt'' and ``Gbb'' respectively.
In the case of the Gtt model, the expected and observed limit at 95\% CL for massless neutralino 
are 2.14 and 1.97 TeV respectively; the main reason of the difference is the excess in SR-0L-HH.
Also for the Gbb model the observed limit is slightly weaker than the expected one: 
while the expected limit for massless neutralino is 2.01 TeV, the observed is 1.92 TeV.
The results of this analysis are reinterpreted also allowing a variable BR of the gluino 
into tt\ninoone, bb\ninoone and tb\chinoonepm. 
These limits, as well as the other limits discussed in this thesis, are obtained with the CLs 
prescription. These results are published in Ref. \cite{Aaboud:2017hrg}. 
The mild excess in SR-0L-HH has been verified also with the 2017 dataset,
leaving the definition of the SRs unchanged; 
together with the 2015 and 2016 data-taking periods, this leads to a total integrated luminosity 
of 79.8 \ifb. 
The update of the analysis did not find an excess in SR-0L-HH, and the increase in luminosity 
allowed to set more stringent limits on gluino pair production. 
In the case of the Gtt model, for neutralino masses below 800 GeV we excluded gluino masses up to 
2.25 TeV, while for the Gbb model the limit is at 2.17 TeV. 
The results of the analysis of the 79.8 \ifb are reinterpreted also for signals with on-shell 
stops. In this case it is found that, while for most m($\tilde{t}$) the sensitivity 
is close to the one obtained in the off-shell case, it becomes weaker 
when m($\tilde{t}$) is close to the kinematic boundary of 
m(\ninoone)+m($t$) and m(\gluino)-m($t$).  
These results have been released in Ref. \cite{ATLAS-CONF-2018-041}. 
While the results included in this thesis are directly related to the ones in Refs. \cite{Aaboud:2017hrg,ATLAS-CONF-2018-041}, 
the high cross-section for gluino pair-production made strong-production multi-$b$ signals among the ones that were targeted since Run 1. 
I have been heavily involved in all the strong-production multi-$b$ results since the beginning of Run 2, in particular the ones in Refs. \cite{Aad:2016eki,ATLAS-CONF-2016-052}. 


The second search targets a General Gauge Mediation (GGM) model of higgsino pair production, where each higgsino then decays to a Higgs boson and 
a gravitino, which in this case is the LSP. 
The search is performed in the channel with four $b$-jets, originating from 
the decay of the two Higgs bosons, and \met. 
This was the first ATLAS analysis targeting this signature, that had been 
previously considered only in searches performed by the CMS collaboration.
This analysis relies on the identification of two Higgs boson candidates in events with at least three or 
at least four $b$-tagged jets. Several orthogonal SRs are optimized and statistically combined in the fit. 
Two discovery SRs are also defined to provide stronger model-independent limits. 
If we assume that the higgsino decays to Higgs boson and gravitino with 100\% BR, 
this analysis excludes \mhino in the range 240-880 GeV at 95\% CL. 
This analysis is complemented by a second analysis targeting signals with low \mhino, where the 
invisible momentum in the final state is not enough to fire the \met trigger. 
Because of a mild excess in this latter analysis, the excluded range of higgsino masses is 
between 130 and 230 GeV and between 290 and 880 GeV.
The results are also interpreted in models with a variable BR of the higgsino into Higgs or $Z$ boson. 
For \mhino = 400 GeV, signal models with BR to Higgs boson higher than 45\% are 
excluded at 95\% CL.
These results are published in Ref. \cite{Aaboud:2018htj}. 


The searched discussed in this thesis are an important element in the wide ATLAS program 
for SUSY searches: they  
provide some of the most restrictive bounds on Natural SUSY scenarios, 
and the experience gained in developing them represents a stepping stone to more sensitive searches 
with the data to be collected in the coming years.

Beside physics analyses, I have also been involved in studies related to the performance of the ATLAS hadronic calorimeter: 
as discussed in Appendix D, I have evaluated the impact of a non-linearity in the response of the photon multiplier tubes of the 
hadronic calorimeter on the measurement of the ATLAS luminosity. 
I have also carried out studies related to the identification of $b$-jets, 
performing studies on the $b$-tagging efficiency for jets with high transverse momentum and contributing to 
the development and validation of a tool to facilitate the use of truth-tagging; this technique 
allows to reduce the statistical uncertainty on samples of simulated events when requiring a high number of $b$-jets. 

%\printbibliography[heading=bibintoc]
\bibliography{two_page_summary}

\end{document}
