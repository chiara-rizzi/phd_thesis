\chapter{LHC and ATLAS}
\label{chap:cern}

The analyses presented in this thesis use the data collected by the ATLAS experiment in 2015, 2016 and 2017 at \cmtre TeV; the ATLAS experiment is one of the four main experiments at the Large Hadron Collider (LHC) at CERN. In this chapter we will first describe the experimental setup of the LHC in sec. \ref{sed:cern:lhc} and then the ATLAS detector in sec. \ref{sed:cern:atlas}.

%%%%%%%%%%%%%%%%%% LHC

\section{The Large Hadron Collider}
\label{sed:cern:lhc}

\subsection{Accelerator Complex}


\subsection{Experiments at the LHC}

\subsubsection*{ALICE}

\subsubsection*{CMS}

\subsubsection*{LHCb}

\subsubsection*{LHCf}

\subsubsection*{TOTEM}

\subsubsection*{ALICE}

\subsubsection*{MoEDAL}


%%%%%%%%%%%%%%%%%% ATLAS

\section{The ATLAS Experiment}
\label{sed:cern:atlas}

\subsection{Coordinate System}

\begin{equation}
\label{eq:cern:eta}
\eta = - \ln \tan \frac{\theta}{2}
\end{equation}

\begin{equation}
\label{eq:cern:y}
y = \frac{1}{2} \ln \frac{E + p_z}{E - p_z}
\end{equation}

\begin{equation}
\label{eq:cern:pt}
p_T = \sqrt{p_x^2 + p_y^2}
\end{equation}

\begin{equation}
\label{eq:cern:pz}
p_z = p_T \,\sinh \eta
\end{equation}

\begin{equation}
\label{eq:cern:dR}
\Delta R = \sqrt{ \Delta \phi^2 + \Delta \eta^2  }
\end{equation}




\subsection{Magnet System}
\label{sec:cern:atlasmagnets}
