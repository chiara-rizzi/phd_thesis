\chapter{LHC and ATLAS}
\label{chap:cern}

The analyses presented in this thesis use the data collected by the ATLAS experiment in 2015, 2016 and 2017 at \cmtre TeV; the ATLAS experiment is one of the four main experiments at the Large Hadron Collider (LHC) at CERN. In this chapter we will first describe the experimental setup of the LHC in sec. \ref{sed:cern:lhc} and then the ATLAS detector in sec. \ref{sed:cern:atlas}.

%%%%%%%%%%%%%%%%%% LHC

\section{The Large Hadron Collider}
\label{sed:cern:lhc}

The momentum of a particle kept in a circular trajectory by a magnetic field is given by:
\begin{equation}
\label{eq:cern:p03br}
p = 0.3 B r,
\end{equation}

\noindent where B is the magnetic field expressed in tesla, r the radius of the circumference in meters and the resulting momentum is expressed in GeV.


Syncrotron Radiation, which leads to an energy loss inversely proportional to the fourth power of the mass of the particle:
\begin{equation}
\label{eq:cern:sync}
\frac{dE}{dt} \propto \frac{E^4}{m^4 r}
\end{equation}


\begin{equation}
\label{eq:cern:nev}
N_{ev} = \sigma \,\, \mathcal{L}_{int}
\end{equation}

The integrated luminosity is the time integral of the instantaneous luminosity $\mathcal{L}$, 

\begin{equation}
\label{eq:cern:intlumi}
\mathcal{L}_{int} = \int \mathcal{L} \, dt
\end{equation}

which in turn is defined as:

\begin{equation}
\label{eq:cern:lumi}
\mathcal{L}=\frac{f n_1 n_2}{4 \pi \sigma_x \sigma_y}
\end{equation}


In eq. \ref{eq:cern:lumi}:
\begin{itemize}
\item f is the revolution frequency
\item $n_1$ and $n_2$ are the number of particle in each beam
\item $\sigma_x$ and $\sigma_y$ are the x and y dimensions of the beam at the interaction point
\end{itemize}

\subsection{Accelerator Complex}

\subsection{Experiments at the LHC}

\subsubsection*{ALICE}

\subsubsection*{CMS}

\subsubsection*{LHCb}

\subsubsection*{LHCf}

\subsubsection*{TOTEM}

\subsubsection*{ALICE}

\subsubsection*{MoEDAL}


%%%%%%%%%%%%%%%%%% ATLAS

\section{The ATLAS Experiment}
\label{sed:cern:atlas}

\subsection{Coordinate System}

\begin{equation}
\label{eq:cern:eta}
\eta = - \ln \tan \frac{\theta}{2}
\end{equation}

\begin{equation}
\label{eq:cern:y}
y = \frac{1}{2} \ln \frac{E + p_z}{E - p_z}
\end{equation}

\begin{equation}
\label{eq:cern:pt}
p_T = \sqrt{p_x^2 + p_y^2}
\end{equation}

\begin{equation}
\label{eq:cern:pz}
p_z = p_T \,\sinh \eta
\end{equation}

\begin{equation}
\label{eq:cern:dR}
\Delta R = \sqrt{ \Delta \phi^2 + \Delta \eta^2  }
\end{equation}




\subsection{Magnet System}
\label{sec:cern:atlasmagnets}

\subsubsection*{Solenoid}

\subsubsection*{Toroids}



\subsection{Inner Detector}



\subsubsection*{Pixel Detector}


\subsubsection*{Semi-Conductor Tracker}


\subsubsection*{Transition Radiation Tracker}



\subsubsection*{IBL}


\subsection{CalorimetersFaculy}

\subsubsection*{Electromagnetic Calorimeter}


\subsubsection*{Hadronic Calorimeter}


\subsubsection*{Forward Calorimeter}


\subsection{Muon Spectrometer}


\subsection{Forward Detectors}


\subsection{ATLAS Performance Summary}


\subsection{ATLAS Physics Program}