\chapter*{Introduction}
\addchaptertocentry{Introduction} 

At the \gls{lhc} \gls{pp} collisions are used to probe the nature of particles at energy regimes that were not accessible before. 
After the discovery of the Higgs boson, the \gls{lhc} is now is now continuing its operations with the dual goal 
of measuring the Higgs boson properties but also to continue in the quest for new particles.

\gls{atlas} is one of the two general purpose experiments on the \gls{lhc} ring, and this thesis focus on the data that is has collected between 
2015 and 2016, in the first part of the \gls{lhc} Run 2, at a center-of-mass energy \cmtre TeV, corresponding to an 
integrated luminosity of approximately 36 \ifb. 

In this thesis I discuss two analyses searching for new physics beyond the standard model in which 
I have been heavily involved 

Top quark as a tool to discover new physics because of the large Yukawa coupling that 

Supersymmetry (SUSY) is an extension of the \gls{sm} of particle physics. 

The content of this thesis is organised as follows. Chapter \ref{chap:SMSUSY} presents an introduction to the \gls{sm}, 
the theory that as of today best describes the experimental results on subatomic particles, moving then to its limits and possible 
extensions, with particular focus on \gls{susy}. 
Chapter \ref{chap:cern} describes the \gls{lhc} accelerator complex, the general techniques used in detectors 
for high-energy physics and the details of the \gls{atlas} detector. 
Chapter \ref{chap:event:MC} discusses the physics of \gls{pp} interactions and how they are simulated with \gls{mc} techniques. 
The event reconstruction and the identification of the physics objects used in the analyses is presented in Chapter \ref{sec:event:reco}. 
Chapter \ref{chap:stat} describes the main statistical procedures used to derive quantitative results in the analysis of the \gls{lhc} data.
Chapter \ref{chap:multib_general} gives a general introduction to the strategies common to both the analyses discussed in this thesis, 
that are finally discussed in Chapter \ref{chap:strong_prod} and Chapter \ref{chap:ewk_prod} for the strong-production and 
electroweak-production analysis respectively. 
Chapter \ref{chap:summary_susy} presents a comparison of the analyses discussed in this thesis with other analyses carried out 
by the \gls{atlas} and \gls{cms} collaborations targeting the similar signal models. 
