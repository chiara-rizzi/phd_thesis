\chapter*{Introduction}
\addchaptertocentry{Introduction} 

At the \gls{lhc} \gls{pp} collisions are used to probe the nature of particles in energy regimes that were not accessible before. 
After the discovery of the Higgs boson in 2012, the \gls{lhc} is now continuing its operations with the dual goal 
of measuring the Higgs boson properties in great detail but also to continue the quest for new particles.

\gls{atlas} is one of the two general purpose experiments on the \gls{lhc} ring, and this thesis focus on the data that is has collected between 
2015 and 2016, 
%in the first part of the \gls{lhc} Run 2, 
at a center-of-mass energy \cmtre TeV, corresponding to an 
integrated luminosity of approximately 36 \ifb. 

The \gls{sm} is the theory that as of today best describes the experimental results on subatomic particles. 
Nevertheless, there are strong theoretical and experimental arguments to believe that the \gls{sm} is the low-energy limit 
of a more general theory, yet to be determined. 
\gls{susy} is one of the most promising \gls{sm} extensions, 
addressing some of its shortcomings. \gls{susy} predicts the existence of partners for  
\gls{sm} particles, which cancel the \gls{sm} corrections to the Higgs boson mass, solving the Naturalness problem. 
In "natural" \gls{susy} models, several particles are expected to be light and therefore observable at the \gls{lhc}.
First of all higgsinos, which share the same tree-level mass parameter as the Higgs boson. 
Then top squarks (stops), that provide a one-loop correction to its mass, and gluinos, that give a two-loop correction since they contribute 
at one-loop at the stop mass. 
The requirement on the stop mass reflects also on the sbottom mass, to which is related through the weak-isospin symmetry. 
These particles are exactly the target of the searches presented in this thesis.
Furthermore, in the framework of \gls{rpcsusy} \gls{susy}, supersymmetric particles are produced in pairs and the 
\gls{lsp} is stable. 
In several models the \gls{lsp} is neutral and weakly interacting, providing a good candidate for dark matter.
 

In this dissertation I discuss two analyses searching for \gls{susy} in final states enriched in $b$-jets, the collimated sprays of 
particles originating from the hadronization of a $b$-quark. 
I have been strongly involved in both analyses, 
and in this thesis I describe with more emphasis the topics where I have given direct contribution.

The first analysis discussed is a search for gluino pair production, where each gluino decays through a stop or a sbottom 
to respectively four top or four bottom quarks and the \gls{lsp}, that is assumed to be neutral and stable.
Since the top quark decays to a $b$-quark and a $W$-boson, both the gluino decay chain through stop and sbottom lead to a 
final state with multiple b-jets and missing transverse momentum (\met). 

The second search targets a \gls{ggm} model of higgsino pair production, where each higgsino then decays to a Higgs boson and 
a gravitino, which in this case is the \gls{lsp}. The high branching ratio of the Higgs boson into a pair of $b$-quarks makes a final state 
rich in $b$-jets promising to tackle this signature.

The content of this dissertation is organized as follows. Chapter \ref{chap:SMSUSY} presents an introduction to the \gls{sm}, 
moving then to its shortcomings and possible extensions, with particular focus on \gls{susy}. 
Chapter \ref{chap:cern} describes the \gls{lhc} accelerator complex, the general techniques used in detectors 
for high-energy physics and the details of the \gls{atlas} detector. 
Chapter \ref{chap:event:MC} discusses the physics of \gls{pp} interactions and how they are simulated with \gls{mc} techniques. 
The event reconstruction and the identification of the physics objects used in the analyses is presented in Chapter \ref{sec:event:reco}. 
Chapter \ref{chap:stat} describes the main statistical procedures used to derive quantitative results in the analysis of the \gls{lhc} data.
Chapter \ref{chap:multib_general} gives a general introduction to the strategies common to both the analyses discussed in this thesis, 
which are discussed in Chapter \ref{chap:strong_prod} and Chapter \ref{chap:ewk_prod} for the gluino search and 
higgsino search respectively. 
Chapter \ref{chap:summary_susy} presents a comparison of the analyses discussed in this thesis with other searches carried out 
by the \gls{atlas} and \gls{cms} collaborations targeting similar signal models. 
Finally, the conclusions are discussed. 

The results presented in this dissertation have lead to the following publications:

\begin{itemize}
\item  ATLAS Collaboration, "Search for Supersymmetry in final states with missing transverse momentum and multiple $b$-jets in proton–proton collisions at \cmtre TeV with the ATLAS detector", JHEP 06 (2018). 
\item  ATLAS Collaboration, "Search for supersymmetry in final states with missing transverse momentum and multiple $b$-jets in proton-proton collisions at \cmtre TeV with the ATLAS detector", ATLAS-CONF-2018-041.
\item  ATLAS Collaboration, "Search for pair production of higgsinos in final states with at least three b-tagged jets in \cmtre TeV pp collisions using the ATLAS detector",  
arXiv:1806.04030 [hep-ex], 
Submitted to: Phys. Rev. (2018).
\end{itemize}

While the paper mention above is the first \gls{atlas} result for the signal model with higgsino pair production, 
the high cross-section for gluino pair-production made strong-production multi-b signals among the ones that were targeted since Run 1. 
I have been heavily involved in all the strong-production multi-b results since the beginning of Run 2, in particular:

\begin{itemize}
\item ATLAS Collaboration, "Search for pair production of gluinos decaying via stop and sbottom in events with $b$-jets and large missing transverse momentum in $pp$ collisions at \cmtre TeV with the ATLAS detector", Phys. Rev. D94.3 (2016).
\item ATLAS Collaboration, "Search for pair production of gluinos decaying via top or bottom squarks in events with $b$-jets and large missing transverse momentum in
$pp$ collisions at \cmtre TeV with the ATLAS detector", ATLAS-CONF-2016-052.
\end{itemize}

Beside physics analyses, I have also been involved in studies related to the performance of the \gls{atlas} hadronic calorimeter: 
as discussed in Appendix \ref{app:pmt}, I have evaluated the impact of a non-linearity in the response of the photon multiplier tubes of the 
hadronic calorimeter on the measurement of the \gls{atlas} luminosity. 
I have also carried out studies related to the identification of $b$-jets, 
performing studies on the $b$-tagging efficiency for jets with high transverse momentum and contributing to 
the development and validation of a tool to facilitate the use of truth-tagging; this technique 
allows to reduce the statistical uncertainty on samples of simulated events when requiring a high number of $b$-jets. 


