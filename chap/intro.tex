\chapter*{Introduction}
\addchaptertocentry{Introduction} 

At the \gls{lhc} proton-proton collisions are used to probe the nature of particles at energy regimes that were not accessible before. 
After the discovery of the Higgs boson, the \gls{lhc} is now is now continuing its operations with the dual goal 
of measuring the Higgs boson properties but also to continue in the quest for new particles.

ATLAS is one of the two general purpose experiments on the \gls{lhc}, and this thesis focus on the data that is has collected between 
2015 and 2016, in the first part of the \gls{lhc} Run 2, at a center-of-mass energy of 13 TeV, corresponding to an 
integrated luminosity of approximately 36 \ifb. 

In this thesis I discuss two analyses searching for new physics beyond the standard model in which 
I have been heavily involved 

Top quark as a tool to discover new physics because of the large Yukawa coupling that 

Supersymmetry (SUSY) is an extension of the \gls{sm} of particle physics. 

The content of this thesis is organised as follows. 
