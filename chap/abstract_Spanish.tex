\chapter*{Resumen}

En esta tesis se presentan dos b\'usquedas de Supersimetr\'ia en colisiones prot\'on-prot\'on en el Gran Colisionador de Hadrones (LHC, por sus siglas en ingl\'es) del CERN,
centradas en modelos que dan lugar a la producci\'on de m\'ultiples quarks top o quarks bottom en el estado final.

La primera b\'usqueda tiene como objetivo la producci\'on de pares de gluinos, donde cada gluino de desintegra a trav\'es de un squark top (modelo Gtt) o un squark  
bottom (modelo Gbb) en una pareja de quarks top-antitop o bottom-antibottom, respectivamente, y un neutralino, que es la part\'icula supersim\'etrica m\'as ligera 
(LSP, por sus siglas en ingl\'es). Cada quak top a su vez se desintegra en un bos\'on $W$ y un quark bottom.
Consecuentemente, el estado final se caracteriza por una alta multiplicidad de bottom jets ($b$-jets), que son chorros de part\'iculas resultantes de
la hadronizaci\'on de quarks bottom, as\'i como un alto momento transverso faltante (\met) debido a que la LSP escapa sin ser detectada. 

La segunda b\'usqueda tiene como objetivo la producci\'on de pares de Higgsinos en el contexto de un modelo GGM, en el cual cada Higgsino 
se desintegra en el bos\'on de Higgs del Model Est\'andar y un gravitino, que en este caso juega el papel de la LSP.
Esta b\'usqueda se centra en sucesos donde ambos bosones de Higgs se desintegran en parejas de quarks bottom-antibottom, de nuevo dado lugar a estados finales 
con m\'ultiples $b$-jets. 
Este es el primer an\'alisis en ATLAS optimizado para esta signatura, la cual hab\'ia sido considerada previamente en b\'usquedas llevadas a cabo por
la colaboraci\'on CMS.

Ambas b\'usquedas en esta tesis usan los datos recopilados por el experimento ATLAS en el LHC 
entre el 2015 y el 2016, a una energ\'ia del centro de masas de \cmtre TeV,
correspondiente a una luminosidad integrada de 36.1 \ifb.
La b\'usqueda de gluinos, sin optimizaci\'on adicional, ha sido extendida usando los datos recopilados en el 2017, resultando en una luminosidad integrada total de 79.8 \ifb.

No habiendo encontrado un exceso significativo de sucesos sobre la predicci\'on del Modelo Est\'andar en ninguna de las regiones de b\'usqueda,
los resultados han sido usados para establecer l\'imites superiores a la producci\'on de part\'iculas supersim\'etricas.
La primera b\'usqueda excluye a un nivel de confianza del 95\% masas del gluino de hasta 2.25 TeV para el modelo Gtt
y de hasta 2.17 TeV para el modelo Gbb, en ambos casos para masas del neutralino menores que 800 GeV.
La segunda b\'usqueda excluye masas del Higgsino en el rango de 240--880 GeV, asumiendo que 
el Higgsino se desintegra exclusivamente en un bos\'on de Higgs y un gravitino.

\par\bigskip
\par\bigskip 
\par\bigskip

\noindent \textbf{Palabras clave}: f\'isica de part\'iculas, CERN, LHC, ATLAS, supersimetr\'ia, nuevos fen\'omenos, b\'usqueda, quark top, bos\'on de Higgs.

