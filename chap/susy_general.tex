\chapter{Common Aspects to Multi-b SUSY Searches}
\label{chap:multib_general}

The main results presented in this thesis are the two analyses described in Chapter \ref{chap:strong_prod} and Chapter \ref{chap:ewk_prod}. While these analyses target two different \gls{susy} models they have many commonalities, since they both target \gls{susy} models leading to final states rich in b-jets and \met. This Chapter highlights these common aspects: Section \ref{sec:simplified_models} describes the philosophy of simplified models, used to design the signal models, Section \ref{sec:common_backgrounds} focuses on the background sources from \gls{sm} processes and how they are modeled in the analyses. 

\section{Simplified Models}
\label{sec:simplified_models}

Even with the simplifying assumptions of the \gls{pmssm}, discussed in Section \ref{sec:theory:pmssm}, a \gls{susy} model has to take into account a large number of free parameters, whose values can impact the characteristics of the particle production and decay. 
Simplified models \cite{Alves:2011wf} are very simple model of \gls{bsm} Physics involving only a few particles and decay mode, 
each one focusing on a specific signature, that can be used to optimize and interpret analyses targeting \gls{bsm} scenarios. 
In general simplified models can be viewed as a limit of more complete models, where all particles except a few are too heavy to be 
produced in the interactions. This leads to a drastic reduction of the number of parameters: a simplified model can be described just by the production cross-section, mass and decay fractions of the few particles considered. 

\section{Analysis Strategy}

\section{Background Processes and their modelling}
\label{sec:common_backgrounds}

This section describes the main background processes from the \gls{sm} in the analysis regions and the way they are 
modeled in the analyses discussed in the next two chapters. In general, the modelling is based on \gls{mc} simulations for
all the backgrounds, except multi-jet which is emulated with a data-driven technique.
For the main background, pair production of top quark pairs, the shape of the different distributions is obtained from \gls{mc}
but the normalization is data-driven, derived in specifically designed \gls{cr} as described in Section \ref{sec:analyses:cr}.
Section \ref{sec:common_obj_def} focuses on the definition of the physic objects, and Section \ref{sec:common_syst} introduces the main sources of 
systematic uncertainties.

\subsection{Top quark pair production}

Top quark pair production constitutes the main source of background in all the analysis regions. 

\subsubsection{Truth-level classification: \ttbar decays}

\subsubsection{Truth-level classification: flavour of the associated jets}

\subsection{Control Regions}
\label{sec:analyses:cr}

\subsection{Single top quark production}

\subsection{Vector boson}

\subsection{Multi-bonson}

\subsection{\ttbar + V production}

\subsection{Multi-jet}


\section{Object Definition}
\label{sec:common_obj_def}

\section{Systematic Uncertainties}
\label{sec:common_syst}

\subsection{Experimental Systematic Uncertainties}

\subsection{Modeling Uncertainties} 

