\chapter{Strong-production multi-b SUSY search}
\label{chap:strong_prod}


\section{Signal model}
\label{sec:strong:signalmodel}

The simplified models used to optimize this analysis are the two gluino-pair-production models shown in Fig. \ref{fig:strong_diagram}. 
Figure \ref{fig:diagram_Gtt} shows a schematic diagram of the Gtt model: 
each of the pair-produced gluinos decays with 100\% \gls{br} to two top quarks and the lightest neutralino (\ninoone).
In the Gbb model, shown in Figure \ref{fig:diagram_Gbb}, the decay happens through an off-shell sbottom and each gluino transforms into 
two bottom quarks and the \ninoone. Both models assume R-parity conservation, so the \ninoone, which in this case is the \gls{lsp}, is stable 
and escapes undetected giving rise to \met in the event. In both cases, the three body decay is realized through an off-shell squark, 
and it involves two interaction vertexes. The first one is a strong-interaction vertex, in the case of the Gtt model:

\begin{equation*}
\gluino \to t \stop^{(*)} \; .
\end{equation*}

\noindent The second is an electroweak vertex, originating from the decay of the \stop (or \sbottom in the case of the Gbb model):

\begin{equation*}
\stop \to t \ninoone
\end{equation*}

\begin{figure*}[h]
\centering 
\subfigure[]{\includegraphics[width=0.35\textwidth]{figures/strong_prod/diagrams/gogo-ttttN1N1.pdf}\label{fig:diagram_Gtt}}
\subfigure[]{\includegraphics[width=0.35\textwidth]{figures/strong_prod/diagrams/gogo-bbbbN1N1.pdf}\label{fig:diagram_Gbb}}
\caption{The simplified models used for the optimization of the analysis. \subref{fig:diagram_Gtt} Gtt model. \subref{fig:diagram_Gbb} Gbb model.
}\label{fig:strong_diagram}
\end{figure*}

In the Gtt and Gbb models, since the \stop and \sbottom are assumed to have very high mass, the only parameters are the mass and production cross section of the gluino, and the mass of the \gls{lsp}. While the Gtt and Gbb models are used both to optimize and interpret the analysis, a further interpretation is provided also in terms of a slightly more complicated model. If we allow also the lightest chargino (\chinoonepm) to have a reachable mass, another decay chain opens, where the virtual stop or sbottom decays to a \chinoonepm:

\begin{equation*}
\begin{split}
\stop &\to \bar{b} \chinoonem \; , \\
\sbottom &\to t \chinoonem \; .
\end{split}
\end{equation*}  

The charge-conjugate processes are also possible, and in both cases the overall decay chain of the gluino leads to $\gluino \to t b \chinoonem$, which we refer to as Gtb model.
The model used to re-interpret this analysis assumes the decay $\chinoonepm \to \ninoone W^{\pm}$, where the W boson can be off-shell if the mass difference between the \chinoonepm and the \ninoone is not large enough to produce an on-shell W boson. 
This mass difference becomes a further parameter of the model, and in this analysis we assume that the \chinoonepm and the \ninoone 
are almost degenerate (in the \gls{mc} simulation, they are generated with a mass difference of 2 GeV); in this case the W boson is virtual and results into soft fermions, that most of the time are not reconstructed. This small mass difference is often verified in models where the \chinoonepm and the \ninoone are part of an approximate $SU(2)$ multiplet, and setting it to a fixed value allows to reduce the number of parameters of the model. 
This analysis provides an interpretation of the result in terms of \gls{br} of the gluino into the Gtt, Gbb and Gtb models. 
A schematic diagram of the possible decay chains in this mixed-\gls{br} interpretation, in addition to the ones shown in Fig. \ref{fig:strong_diagram}, is shown in Fig. \ref{fig:strong_diagram_br}.

\begin{figure*}[h]
\centering 
\subfigure[]{\includegraphics[width=0.35\textwidth]{figures/strong_prod/diagrams/gogo-tbfftbffN1N1.pdf}\label{fig:diagram_strong_tbtb}}
\subfigure[]{\includegraphics[width=0.35\textwidth]{figures/strong_prod/diagrams/gogo-tttbffN1N1.pdf}\label{fig:diagram_strong_tttb}}\\
\subfigure[]{\includegraphics[width=0.35\textwidth]{figures/strong_prod/diagrams/gogo-bbtbffN1N1.pdf}\label{fig:diagram_strong_bbtb}}
\subfigure[]{\includegraphics[width=0.35\textwidth]{figures/strong_prod/diagrams/gogo-ttbbN1N1.pdf}\label{fig:diagram_strong_ttbb}}
\caption{Simplified models used in the re-interpretation of the analysis. \subref{fig:diagram_strong_tbtb} Both gluinos have the following decay chain: $\gluino \to t \bar{b} \chinoonem$ with $\chinoonem \to f\bar{f}' \ninoone$. \subref{fig:diagram_strong_tttb} One gluino decays as in \subref{fig:diagram_strong_tbtb} and the other as $\gluino \to t\bar{t}\ninoone$. \subref{fig:diagram_strong_bbtb} One gluino decays as in \subref{fig:diagram_strong_tbtb} and the other as $\gluino \to b\bar{b}\ninoone$. \subref{fig:diagram_strong_ttbb} One gluino decays as $\gluino \to t\bar{t}\ninoone$ and the other as $\gluino \to b\bar{b}\ninoone$.  %The charge conjugate processes are implied.
%The fermions originating from the $\chinoonepm$ decay are typically soft because the mass difference  between the $\chinoonepm$ and the $\ninoone$ is fixed to 2 GeV. 
}\label{fig:strong_diagram_br}
\end{figure*}


These gluino decay chains that lead to final states rich in heavy-flavour quarks are dominant in \gls{susy} models where the squarks 
from the first two generations are significantly heavier than stop and sbottom, or also in cases when the \ninoone is dominated by the higgsino component. 

%\section{Previous limits}

\section{Kinematic variables specific to this analysis}
\label{sec:variables-strong}

Most of the kinematic variables used in this analysis to discriminate between signal and background 
are in common with the electroweak-production analysis and have already been discussed in Section \ref{sec:common_variables}.
Two other variables, used only in this analysis, are introduced in this section.


\subsubsection*{Total jet mass}

The total jet mass is the sum of the masses of the four leading reclustered jets in the event:

\begin{equation}
\mjsum = \sum_{i \leq 4} m_{J,i} \; .
\end{equation}

\noindent This variable is particularly useful in the case of boosted Gtt signals, where the four top quarks are produced with high momentum and can be reconstructed as re-clustered jets with high mass. 

\subsubsection*{ISR jet}

In the case of Gbb models, when the mass of the neutralino is close to the mass of the gluino the neutralino is produced almost at rest and the event does not have a sizable \met; this reduces the sensitivity of the analysis to these cases. 
Also for these signal samples, a sizable value of \met can be present in events where the gluino system is recoiling against a hard jet from \gls{isr}; if we assume that the \gls{isr} jet is the most energetic jet in the event, there will be an angular separation between this jet and \met larger than the average. The variable \dphilead, defined as:

\begin{equation}
\dphilead = |{\Delta}{\phi}(j_1, \met)| \; ,
\end{equation}

\noindent helps therefore in selecting signal events where the \met derives from a neutralino boosted by an \gls{isr} jet. 


\section{Discriminating variables}
\label{sec:strong:sigbkg}

In this section we show how the variables defined in Section \ref{sec:common_variables} and \ref{sec:variables-strong} allow to 
identify a region of the phase space that is enriched in signal events. 
After selecting events with high \met ($> 200$ GeV), at least four signal jets and at least three b-tagged jets, events are
divided into the 0-lepton category, that requires a lepton veto and also \dphimin $>0.4$, and the 1-lepton category, 
with at least one signal lepton.
Figures \ref{fig:strong:sig:bjets_n} to \ref{fig:strong:sig:met} show the distribution of some important kinematic variables for 
the sum of the \gls{sm} backgrounds and for selected signal samples. 

From Figure \ref{fig:strong:sig:bjets_n} it is possible to see how signal events have in general higher number of 
b-tagged jets than background events; despite this, this variable alone is not enough to reduce the number of 
background events enough to be sensitive to the signal.

Figure \ref{fig:strong:sig:jets_n} shows the distribution of the number of signal jets. 
In this case, the result of the comparison between the shape of the distribution for signal and for backgrounds depends heavily 
on the specific signal considered. In particular, Gbb signal models tend to have a lower number of jets than Gtt models; 
this is expected, since in Gtt models the decay of the top quark to a b-quark and a W-boson can lead to up to 12 jets originating from the 
hard scattering process, while only four of such jest are present in Gbb signals. 
Furthermore, Gtt events in the 0-lepton channel have more jets than Gttt-events in the 1-lepton channel, since the leptpnic decays of the W-boson 
reduce the amount of hadronic activity in the final state. 
It can be noted how while in \gls{sm} simulations all-hadronic \ttbar events do not enter our analysis regions because of the lack of real \met,  in signal events we can have events where all the top quarks decay hadronically, since the \met is provided by the neutralino. 

A key variable in the suppression of the \ttbar background is \mtb, whose distribution is shown in Figure \ref{fig:strong:sig:mTb_min}.
The kinematic endpoint that characterizes \ttbar events is not present for signal events, that tend to have a more uniform distribution.
Despite not having a kinematic endpoint, not all signals have a \mtb distribution that reaches high values: this is more likely for 
signals with a high mass splitting between the gluino and the neutralino masses, where the decay products are boosted.

The \meff distribution for signal and background events is shown in Figure \ref{fig:strong:sig:meff_incl}. 
In this case it is possible to observe how signal models with a low neutralino mass (and therefore a large mass splitting)
this variable can be extremely helpful in identifying the presence of signal. 
Instead \meff does not offer a large discriminating power for signals that originate less energetic final states, as it is the 
case for the region of the parameter space where the mass of the neutralino approaches the mass of the gluino.
A similar argument can be made for \mjsum and \met, shown in Figures \ref{fig:strong:sig:MJSum_rc_r08pt10} and \ref{fig:strong:sig:met} respectively. 


\begin{figure*}[h]
\centering 
\subfigure[]{\includegraphics[width=0.325\textwidth]{figures/strong_prod/sig_bkg_strong/1L_3b/Gtt_compare_bjets_n.pdf}\label{fig:strong:sig:bjets_nA}}
\subfigure[]{\includegraphics[width=0.325\textwidth]{figures/strong_prod/sig_bkg_strong/0L_3b/Gtt_compare_bjets_n.pdf}\label{fig:strong:sig:bjets_nB}}
\subfigure[]{\includegraphics[width=0.325\textwidth]{figures/strong_prod/sig_bkg_strong/0L_3b/Gbb_compare_bjets_n.pdf}\label{fig:strong:sig:bjets_nC}}
\caption{Distribution of number of b-tagged jets in background events and in \subref{fig:strong:sig:bjets_nA} Gtt signals in a 1-lepton selection,
\subref{fig:strong:sig:bjets_nB} Gtt signals in a 0-lepton selection and
\subref{fig:strong:sig:bjets_nC} Gbb signals in a 0-lepton selection.
}\label{fig:strong:sig:bjets_n}
\end{figure*}

\begin{figure*}[h]
\centering 
\subfigure[]{\includegraphics[width=0.325\textwidth]{figures/strong_prod/sig_bkg_strong/1L_3b/Gtt_compare_jets_n.pdf}\label{fig:strong:sig:jets_nA}}
\subfigure[]{\includegraphics[width=0.325\textwidth]{figures/strong_prod/sig_bkg_strong/0L_3b/Gtt_compare_jets_n.pdf}\label{fig:strong:sig:jets_nB}}
\subfigure[]{\includegraphics[width=0.325\textwidth]{figures/strong_prod/sig_bkg_strong/0L_3b/Gbb_compare_jets_n.pdf}\label{fig:strong:sig:jets_nC}}
\caption{Distribution of number of jets in background events and in \subref{fig:strong:sig:jets_nA} Gtt signals in a 1-lepton selection,
\subref{fig:strong:sig:jets_nB} Gtt signals in a 0-lepton selection and
\subref{fig:strong:sig:jets_nC} Gbb signals in a 0-lepton selection.
}\label{fig:strong:sig:jets_n}
\end{figure*}

\begin{figure*}[h]
\centering 
\subfigure[]{\includegraphics[width=0.325\textwidth]{figures/strong_prod/sig_bkg_strong/1L_3b/Gtt_compare_mTb_min.pdf}\label{fig:strong:sig:mTb_minA}}
\subfigure[]{\includegraphics[width=0.325\textwidth]{figures/strong_prod/sig_bkg_strong/0L_3b/Gtt_compare_mTb_min.pdf}\label{fig:strong:sig:mTb_minB}}
\subfigure[]{\includegraphics[width=0.325\textwidth]{figures/strong_prod/sig_bkg_strong/0L_3b/Gbb_compare_mTb_min.pdf}\label{fig:strong:sig:mTb_minC}}
\caption{Distribution of \mtb in  background events and in \subref{fig:strong:sig:mTb_minA} Gtt signals in a 1-lepton selection, 
\subref{fig:strong:sig:mTb_minB} Gtt signals in a 0-lepton selection and \subref{fig:strong:sig:mTb_minC} Gbb signals in a 0-lepton selection.
}\label{fig:strong:sig:mTb_min}
\end{figure*}

\begin{figure*}[h]
\centering 
\subfigure[]{\includegraphics[width=0.325\textwidth]{figures/strong_prod/sig_bkg_strong/1L_3b/Gtt_compare_meff_incl.pdf}\label{fig:strong:sig:meff_inclA}}
\subfigure[]{\includegraphics[width=0.325\textwidth]{figures/strong_prod/sig_bkg_strong/0L_3b/Gtt_compare_meff_incl.pdf}\label{fig:strong:sig:meff_inclB}}
\subfigure[]{\includegraphics[width=0.325\textwidth]{figures/strong_prod/sig_bkg_strong/0L_3b/Gbb_compare_meff_incl.pdf}\label{fig:strong:sig:meff_inclC}}
\caption{Distribution of \meff in background events and in \subref{fig:strong:sig:meff_inclA} Gtt signals in a 1-lepton selection, \subref{fig:strong:sig:meff_inclB} Gtt signals in a 0-lepton selection and \subref{fig:strong:sig:meff_inclC} Gbb signals in a 0-lepton selection.
}\label{fig:strong:sig:meff_incl}
\end{figure*}


\begin{figure*}[h]
\centering 
\subfigure[]{\includegraphics[width=0.325\textwidth]{figures/strong_prod/sig_bkg_strong/1L_3b/Gtt_compare_MJSum_rc_r08pt10.pdf}\label{fig:strong:sig:MJSum_rc_r08pt10A}}
\subfigure[]{\includegraphics[width=0.325\textwidth]{figures/strong_prod/sig_bkg_strong/0L_3b/Gtt_compare_MJSum_rc_r08pt10.pdf}\label{fig:strong:sig:MJSum_rc_r08pt10B}}
\subfigure[]{\includegraphics[width=0.325\textwidth]{figures/strong_prod/sig_bkg_strong/0L_3b/Gbb_compare_MJSum_rc_r08pt10.pdf}\label{fig:strong:sig:MJSum_rc_r08pt10C}}
\caption{Distribution of \mjsum in background events and in \subref{fig:strong:sig:MJSum_rc_r08pt10A} Gtt signals in a 1-lepton selection, 
\subref{fig:strong:sig:MJSum_rc_r08pt10B} Gtt signals in a  0-lepton selection and
\subref{fig:strong:sig:MJSum_rc_r08pt10C} Gbb signals in a  0-lepton selection.
}\label{fig:strong:sig:MJSum_rc_r08pt10}
\end{figure*}


\begin{figure*}[h]
\centering 
\subfigure[]{\includegraphics[width=0.325\textwidth]{figures/strong_prod/sig_bkg_strong/1L_3b/Gtt_compare_met.pdf}\label{fig:strong:sig:metA}}
\subfigure[]{\includegraphics[width=0.325\textwidth]{figures/strong_prod/sig_bkg_strong/0L_3b/Gtt_compare_met.pdf}\label{fig:strong:sig:metB}}
\subfigure[]{\includegraphics[width=0.325\textwidth]{figures/strong_prod/sig_bkg_strong/0L_3b/Gbb_compare_met.pdf}\label{fig:strong:sig:metC}}
\caption{Distribution of \met in  background events and in \subref{fig:strong:sig:metA} Gtt signals in a 1-lepton selection, 
\subref{fig:strong:sig:metB} Gtt signals in a  0-lepton selection
 and \subref{fig:strong:sig:metC} Gbb signals in a  0-lepton selection.
}\label{fig:strong:sig:met}
\end{figure*}



\section{Cut-and-count analysis regions}
\label{sec:strong:cutandcount}
In this section, we discuss the definition of the cut-and-count analysis regions: 
first of all the optimization process that leads to the \glspl{sr} and then the 
design of corresponding \glspl{cr} for the normalization of the \ttbar background and 
\glspl{vr} to test the validity of the background prediction.

\subsection*{Signal regions}

As discussed in Section \ref{sec:analysisstrategy}, the cut-and-count \glspl{sr} are designed to maximize the significance to specific 
signal benchmarks, where the significance is defined as the number of standard deviations in a Gaussian that give the p-value obtained with Eq. \ref{eq:binomexpp}.
Since they are not meant to be statistically combined, they do not need to be mutually exclusive and each of them can be optimized independently.
The variables considered in the optimization are chosen by selecting the ones that show the most vivid differences in shape between signal and background, and are \njet, \nbjet, \met, \meff, \mjsum, \mtb and, in the case of regions requiring at least one lepton, \mt.
The \glspl{sr} are define as the set of selections that maximize the expected significance for each benchmark model while fulfilling these requirements:
\begin{itemize}
\item At least 0.5 expected background events.
\item \ttbar as major background component.
\item \gls{mc} statistical uncertainty on the \ttbar component $<30$\%.
\item At least 2 expected events for the benchmark signal considered. 
\end{itemize}
After selecting events with at lest four jet, at least three b-jets, that fired the \met trigger and with $\met>200$ GeV, three separate optimizations are performed:

\begin{description}

\item[Gtt-1L] A first set of regions targets the Gtt signals (shown in Fig. \ref{fig:diagram_Gtt}) in the final states with at least one reconstructed lepton. Three different \glspl{sr} are defined in this category: boosted (B), targeting signal models with high gluino mass 
and a large mass difference between the gluino and the neutralino, medium (M), targeting signal models with intermediate mass splitting, and compressed (C), aiming at signal models where the mass of the neutralino is close to the mass of the gluino. 

\item[Gtt-0L] A separate optimization is performed for the cases where the Gtt model leads to final states where there are no reconstructed leptons.
Also in this case, three \glspl{sr} are defined and referred to as B, M and C depending on the signal models they are targeting. 
All the Gtt \glspl{sr} are reported in Table \ref{tab:GttEvsel}, together with their \glspl{cr} and \glspl{sr} whose design is discussed in the next section. 

\item[Gbb] The \glspl{sr} for the Gbb model (shown in Fig. \ref{fig:diagram_Gbb}) is performed separately. 
This signal does not produce any prompt lepton in the final state, but the characteristics of this signal differ noticeably form 
the case of Gtt-0L, particularly for the lower number of jets. 
Four \glspl{sr} are optimized for the Gbb signals: boosted (B), medium (M), compressed (C) and very compressed (VC). 
The last \gls{sr} targets compressed Gbb models, that fire the \met trigger because the gluino system is recoiling against 
and \gls{isr} jet, giving additional boost to the neutralinos. To enhance this topology, this region requires that the leading jet is b-tagged, and 
the variable \dphilead is included in the optimization. The selections resulting from the optimization of the Gbb \glspl{sr} are 
reported in Table \ref{tab:Gbb0LEvsel}, together with the corresponding \glspl{cr} and \glspl{vr}.

\end{description}

\subsubsection*{Control and validation regions}

All the \glspl{cr} of the cut-and-count regions, including the ones for the Gbb \glspl{sr} and for the Gtt-1L \glspl{sr}
require the presence of at least one signal lepton. 
The orthogonality with the Gtt-1L \glspl{sr} is ensured by applying to all the \glspl{cr} an upper selection on $\mt < 150$ GeV.
To have enough statistics in the \glspl{cr} to properly constrain the \ttbar background, the 
\mtb selection is removed and other selections are relaxed to ensure at least 10 expected background events in each \gls{cr}.

In the Gtt-1L regions the extrapolation between each pair of \gls{cr} and \gls{sr} is validated in two different \glspl{vr}:
\gls{vr}-\mt is designed to verify the extrapolation to high \mt, and is maintained orthogonal to the \gls{sr} through an inverted \mjsum 
selection. A second region, \gls{vr}-\mtb, tests the high-\mtb extrapolation by selecting events with high \mtb and low \mt; 
this region is orthogonal to the corresponding \gls{cr} thanks to the exclusive jet multiplicity requirement that characterizes
the Gtt-1L \glspl{cr}.

In the case of the Gtt-0L regions, the main extrapolation between each pair of \gls{cr} and \gls{sr} is on the number of leptons. 
This is validated in a specifically-designed 0-lepton \gls{vr}, 
that is kept orthogonal to the corresponding \gls{sr} by an inverted \mjsum selection.

Similarly, also each Gbb region has a 0-lepton \gls{vr}, whose orthogonality with the \gls{sr} is maintained through 
a shift in the \meff selection for regions Gbb-B and Gbb-M, and in the \met selection in regions Gbb-C and Gbb-VC (since these last 
two regions target signal models with a compressed mass spectrum, and therefore do not apply any \meff selection in the \gls{sr}).



\begin{table}[t]
    \centering
 \renewcommand{\arraystretch}{1.5}
         \begin{tabular}{c c c c c c c c}
        \toprule
\multicolumn{8}{c}{\textbf{ Gtt-1L}}\\
\multicolumn{8}{c}{Criteria common to all regions: $\ge 1$ signal lepton, ${\pt}^\mathrm{jet} >  30~\gev$, $\nbjet \geq 3$} \\\midrule
Targeted kinematics & Type & $\njet$ & $\mt$ & $\mtb$& $\met$ & $\meffi$ & $\mjsum$ \\ \midrule
\multirow{4}{*}{\begin{minipage}{3cm}\centering Region B\\
              (Boosted, Large \msplit) \end{minipage}} 
 & SR & $\ge 5$ & $> 150$ & $> 120 $  & $> 500 $ & $> 2200 $ & $> 200$  \\
 & CR & $= 5$ & $< 150$ & $-$  & $> 300 $ & $> 1700 $ & $> 150$  \\
 & VR-$\mt$ & $\ge 5$ & $> 150$ & $-$  & $> 300 $ & $> 1600 $ & $< 200$  \\
& VR-$\mtb$ & $> 5$ & $< 150$ & $> 120 $  & $> 400 $ & $> 1400 $ & $> 200$  \\\midrule
\multirow{4}{*}{\begin{minipage}{3cm}\centering Region M\\
              (Moderate \msplit) \end{minipage}} 
 & SR & $\ge 6$ & $> 150$ & $> 160 $  & $> 450 $ & $> 1800 $ & $> 200$  \\
 & CR & $= 6$ & $< 150$ & $-$  & $> 400 $ & $> 1500 $ & $> 100$  \\
 & VR-$\mt$ & $\ge 6$ & $> 200$ & $-$  & $> 250 $ & $> 1200 $ & $< 100$  \\
& VR-$\mtb$ & $> 6$ & $< 150$ & $> 140 $  & $> 350 $ & $> 1200 $ & $> 150$  \\\midrule
\multirow{4}{*}{\begin{minipage}{3cm}\centering Region C\\
              (Compressed, small \msplit) \end{minipage}} 
 & SR & $\ge 7$ & $> 150$ & $> 160 $  & $> 350 $ & $> 1000 $ & $-$  \\
 & CR & $= 7$ & $< 150$ & $-$  & $> 350 $ & $> 1000 $ & $-$  \\
 & VR-$\mt$ & $\ge 7$ & $> 150$ & $< 160 $  & $> 300 $ & $> 1000 $ & $-$  \\
& VR-$\mtb$ & $> 7$ & $< 150$ & $> 160 $  & $> 300 $ & $> 1000 $ & $-$  \\
      \end{tabular}
         \begin{tabular}{c c c c c c c c c c c}
        \toprule
\multicolumn{11}{c}{\textbf{ Gtt-0L}}\\
\multicolumn{11}{c}{Criteria common to all regions: ${\pt}^\mathrm{jet} > 30$~GeV} \\\midrule
Targeted kinematics & Type & $N_\mathrm{lepton}$ & $\nbjet$& $\njet$&  $\dphimin$ & $\mt$ & $\mtb$ & $\met$ & $\meffi$ & $\mjsum$ \\ \midrule
\multirow{3}{*}{\begin{minipage}{3cm}\centering Region B\\
              (Boosted, Large \msplit) \end{minipage}} 
& SR & $= 0$  & $\ge 3$ & $\ge 7$ & $>0.4$ & $-$ & $> 60 $ & $> 350 $ & $> 2600$ & $> 300$\\ 
& CR & $= 1$  & $\ge 3$ & $\ge 6$ & $-$ & $<150$ & $-$ & $> 275 $ & $> 1800$ & $> 300$\\ 
& VR & $= 0$  & $\ge 3$ & $\ge 6$ & $>0.4$ & $-$ & $-$ & $> 250 $ & $> 2000$ & $< 300$\\ \midrule
\multirow{3}{*}{\begin{minipage}{3cm}\centering Region M\\
              (Moderate \msplit) \end{minipage}} 
& SR & $= 0$  & $\ge 3$ & $\ge 7$ & $>0.4$ & $-$ & $> 120 $ & $> 500 $ & $> 1800$ & $> 200$\\ 
& CR & $= 1$  & $\ge 3$ & $\ge 6$ & $-$ & $<150$ & $-$ & $> 400 $ & $> 1700$ & $> 200$\\ 
& VR & $= 0$  & $\ge 3$ & $\ge 6$ & $>0.4$ & $-$ & $-$ & $> 450 $ & $> 1400$ & $< 200$\\ \midrule
\multirow{3}{*}{\begin{minipage}{3cm}\centering Region C\\
              (Compressed, moderate \msplit) \end{minipage}} 
& SR & $= 0$  & $\ge 4$ & $\ge 8$ & $>0.4$ & $-$ & $> 120 $ & $> 250 $ & $> 1000$ & $> 100$\\ 
& CR & $= 1$  & $\ge 4$ & $\ge 7$ & $-$ & $<150$ & $-$ & $> 250 $ & $> 1000$ & $> 100$\\ 
& VR & $= 0$  & $\ge 4$ & $\ge 7$ & $>0.4$ & $-$ & $-$ & $> 250 $ & $> 1000$ & $< 100$\\ 

\bottomrule
\end{tabular}
\caption{Definitions of the Gtt SRs, CRs and VRs of the cut-and-count analysis.  All kinematic variables are
   expressed in \gev\ except $\dphimin$, which is in radians. The jet \pt\ requirement is also applied to 
   $b$-tagged jets.}
      \label{tab:GttEvsel}
 \end{table}

%\clearpage


\begin{landscape}
\begin{table}[t]
%\begin{sidewaystable}[t]
    \centering
 \renewcommand{\arraystretch}{1.3}
         \begin{tabular}{c c c c c c c c c c}
        \toprule
\multicolumn{10}{c}{\textbf{ Gbb}}\\
\multicolumn{10}{c}{Criteria common to all regions: $\njet \geq 4$,
           ${\pt}^\mathrm{jet} > 30$~GeV } \\\midrule 
Targeted kinematics  & Type & $N_\mathrm{lepton}$ & $\nbjet$ &  $\dphimin$ & $\mt$ & $\mtb$ & $\met$ & $\meff$ & Others  \\\midrule
\multirow{3}{*}{\begin{minipage}{3cm}\centering Region B\\
              (Boosted, Large \msplit) \end{minipage}} 
& SR & $= 0$  & $\ge 3$ & $>0.4$ & $-$ & $- $ & $> 400 $ & $> 2800$ & $-$ \\ 
& CR & $= 1$  & $\ge 3$ & $-$ & $< 150$ & $- $ & $> 400 $ & $> 2500$ & $-$ \\ 
& VR & $= 0$  & $\ge 3$ & $>0.4$ & $-$ & $- $ & $> 350 $ & $1900$--$2800$ & $-$ \\\midrule
\multirow{3}{*}{\begin{minipage}{3cm}\centering Region M\\
              (Moderate \msplit) \end{minipage}} 
& SR & $= 0$  & $\ge 4$ & $>0.4$ & $-$ & $>90$ & $> 450 $ & $> 1600$ & $-$ \\ 
& CR & $= 1$  & $\ge 4$ & $-$ & $< 150$ & $- $ & $> 300 $ & $> 1600$ & $-$ \\ 
& VR & $= 0$  & $\ge 4$ & $>0.4$ & $-$ & $>100$ & $250$--$450$ & $1600$--$1900$ & $-$ \\\midrule
\multirow{3}{*}{\begin{minipage}{3cm}\centering Region C\\
              (Compressed, small \msplit) \end{minipage}} 
& SR & $= 0$  & $\ge 4$ & $>0.4$ & $-$ & $>155$ & $> 450 $ & $-$ & $-$ \\ 
& CR & $= 1$  & $\ge 4$ & $-$ & $< 150$ & $- $ & $> 375 $ & $-$ & $-$ \\ 
& VR & $= 0$  & $\ge 4$ & $>0.4$ & $-$ & $>125$ & $350$--$450$ & $-$ & $-$ \\\midrule
\multirow{3}{*}{\begin{minipage}{3cm}\centering Region VC\\
              (Very Compressed, very small \msplit) \end{minipage}} 
& SR & $= 0$  & $\ge 3$ & $>0.4$ & $-$ & $>100$ & $> 600 $ & $-$ &
                                                                   \multirow{3}{*}{\begin{minipage}{3cm}\centering $\pt^{\leadjet}>400$, $\leadjet \neq b$, $\dphilead>2.5$\end{minipage}} \\ 
& CR & $= 1$  & $\ge 3$ & $-$ & $< 150$ & $- $ & $> 600 $ & $-$ \\ 
& VR & $= 0$  & $\ge 3$ & $>0.4$ & $-$ & $>100$ & $225$--$600$ & $-$ \\
      \bottomrule
    \end{tabular}
      \caption{Definitions of the Gbb SRs, CRs and VRs of the cut-and-count analysis.  
  All kinematic variables are expressed in \gev\ except $\dphimin$, which is in radians.
   The jet \pt\ requirement is applied to the 
   four leading jets, a subset of which are $b$-tagged jets. 
   The $\leadjet \neq b$  requirement specifies that the leading jet is not $b$-tagged.
   }
       \label{tab:Gbb0LEvsel}
%\end{sidewaystable}
 \end{table}
\end{landscape}



%\clearpage

\subsubsection*{Composition of the cut-and-count regions}

The pre-fit background composition of the cut-and-count analysis regions is show in Figures \ref{fig:bkgcomp_Gtt1L} to \ref{fig:bkgcomp_Gbb}.
It is possible to see how \ttbar is the dominant background in all the \glspl{sr}; the \glspl{cr} and the \glspl{vr} have a high \ttbar 
purity by construction, since they are designed to respectively normalize this background and validate the extrapolation of this normalization to 
a phase space closer to the \glspl{sr}.

\begin{figure}[h]
\includegraphics[width=\textwidth]{figures/strong_prod/comp_plots/Gtt_1L_bkg.pdf}
\caption{Background composition of Gtt-1L regions.}
	\label{fig:bkgcomp_Gtt1L}
\end{figure}

\begin{figure}[h]
\includegraphics[width=\textwidth]{figures/strong_prod/comp_plots/Gtt_0L_bkg.pdf}
\caption{Background composition of Gtt-0L regions.}
	\label{fig:bkgcomp_Gtt0L}
\end{figure}

\begin{figure}[h]
\includegraphics[width=\textwidth]{figures/strong_prod/comp_plots/Gbb_bkg.pdf}
\caption{Background composition of Gbb regions.}
	\label{fig:bkgcomp_Gbb}
\end{figure}

%%%

Figures \ref{ffig:bkgcomp_Gtt1L} to \ref{fig:bkgcomp_Gbb} show the decay type of the \ttbar background for  Gtt-1L, Gtt-0L and Gbb regions respectively, classified as described in Section \ref{sec:susy_general:ttbar}.
Looking at Figure \ref{fig:Gtt_1L_tt} we can see that the \glspl{cr} are dominated by single-lepton \ttbar, as well as the \glspl{vr} designed to check the \mtb extrapolation;
this is because both these types of regions have an upper selection on \mt.
Instead the \glspl{sr} and the \glspl{vr} that check the \mt extrapolation occupy a phase space at high-\mt;
this selection suppresses single-lepton \ttbar and therefore the \glspl{sr} and \glspl{vr}-\mt
are dominated by dilepton \ttbar.
In the case of Figures \ref{fig:bkgcomp_Gtt0L} and \ref{fig:bkgcomp_Gbb} we see all regions are dominated by single-lepton \ttbar.
In these cases, the \glspl{cr} are 1-lepton regions with an upper cut on \mt and the \glspl{sr} and \glspl{vr} are 0-lepton regions,
so in all of them the main component is  single-lepton \ttbar. In the case of SRs and VRs though, this single-lepton component
is dominated by $\tau$+jets, since this category includes both hadronic and leptonic decays of the tau lepton.


\begin{figure}[h]
\includegraphics[width=\textwidth]{figures/strong_prod/comp_plots/Gtt_1L_tt.pdf}
\caption{Decay mode of \ttbar background in Gtt-1L regions.}
	\label{fig:ttcomp_Gtt1L}
\end{figure}

\begin{figure}[h]
\includegraphics[width=\textwidth]{figures/strong_prod/comp_plots/Gtt_0L_tt.pdf}
\caption{Decay mode of \ttbar background in Gtt-0L regions.}
	\label{fig:ttcomp_Gtt0L}
\end{figure}

\begin{figure}[h]
\includegraphics[width=\textwidth]{figures/strong_prod/comp_plots/Gbb_tt.pdf}
\caption{Decay mode of \ttbar background in Gbb regions.}
	\label{fig:ttcomp_Gbb}
\end{figure}

%%%%

The classification of the \ttbar background based on the flavour of the jets associated to the 
\ttbar production, as described in Section \ref{sec:susy_general:ttbar}, is shown in Figures 
\ref{fig:HFcomp_Gtt1L} to \ref{fig:HFcomp_Gbb}.
Thanks to adopting the same selection on the number of b-jets in \gls{cr}, \gls{sr} and \glspl{vr} 
of the same region type, composition is similar in corresponding regions, even if the fraction of 
\tthf is sometimes higher in the \glspl{sr}, especially in the case of Gtt-1L regions. 


\begin{figure}[h]
\includegraphics[width=\textwidth]{figures/strong_prod/comp_plots/Gtt_1L_HF.pdf}
\caption{Heavy-flavour composition of \ttbar background in Gtt-1L regions.}
	\label{fig:HFcomp_Gtt1L}
\end{figure}

\begin{figure}[h]
\includegraphics[width=\textwidth]{figures/strong_prod/comp_plots/Gtt_0L_HF.pdf}
\caption{Heavy-flavour composition of \ttbar background in Gtt-0L regions.}
	\label{fig:HFcomp_Gtt0L}
\end{figure}

\begin{figure}[h]
\includegraphics[width=\textwidth]{figures/strong_prod/comp_plots/Gbb_HF.pdf}
\caption{Heavy-flavour composition of \ttbar background in Gbb regions.}
	\label{fig:HFcomp_Gbb}
\end{figure}

%%%%%

\clearpage

\section{Multi-bin analysis regions}
\label{sec:strong:multibin}
In this section we describe the definition of the \glspl{sr}, \gls{cr} and \gls{vr} 
of the multi-bin analysis. 

\subsection*{Signal regions}

The goal of the multi-bin strategy is to provide a series of regions each optimized for a different signal model 
but all mutually exclusive, so that they can be statistically combined to increase the exclusion power of the analysis.

Figure \ref{fig:strong:sig:jets_n} shows that the number of jets provides a good handle to separate 
between Gbb signal models, with lower number of jets, and the Gtt signal models, where the number of jets is instead higher;
the Gtb model described in Section \ref{sec:strong:signalmodel} will have a number of jets intermediate between the Gtt and the Gtb case, 
as well as the mixed-\gls{br} case where one of the two produced gluinos decays to $b \bar{b} \ninoone$ and the other to 
$t \bar{t} \ninoone$.
Within a single decay topology, the variable that discriminates the best between signals with different mass splitting between $\tilde{g}$ and
\ninoone is \meff: as shown in Figure \ref{fig:strong:sig:meff_incl}, signal models with larger mass splitting tend to have higher values 
of \meff, since the decay products are more boosted. 

These two variables are therefore used, together with the number of signal leptons, to slice the phase space into mutually exclusive regions. 
A schematic view of this slicing is given in Figure \ref{fig:multibin_scheme}, while the precise numerical values can be found 
in tables \ref{tab:multibin_Hn} to \ref{tab:multibin_Ln}.
The naming convention for the multi-bin \glspl{sr} is the sequence of number of leptons, category for number of jets and category for \meff regime,
where the categories for \njet and \meff are labeled as "L" (low), "I" (intermediate) and "H" (high). So e.g. SR-0L-IL is the \gls{sr} in the 0-lepton channel with intermediate number of jets and low \meff. Low \meff regions are defined only in the 0-lepton channel, since they 
target Gbb signal models that do not produce final states with leptons. 

In each of these mutually exclusive \gls{sr}, all the variables other than \njet, \nlep and \meff are optimized in a two-steps procedure:
\begin{enumerate}
\item First of all, in each region we define the set of selection that maximize the expected significance for a specific benchmark model. 
These selections have to satisfy the same criteria as for the cut-and-count \glspl{sr}, described in Section \ref{sec:strong:cutandcount}.
\item In  second step, we try to make the selections more uniform across all the regions, without penalizing the exclusion power. 
For each desired simplification of the selection, the exclusion contour is computed and compared to the original one obtained with the selections from the first step. If the simplification of the selections does not bring any significant loss in the exclusion sensitivity, the simplification is adopted. This is the case e.g. for the \mt selection, which takes the value of $>150$ GeV for all the 1-lepton \glspl{sr}.
\end{enumerate}


\begin{figure}[h]
	\subfigure[]{\includegraphics[width=0.49\linewidth]{figures/strong_prod/paper/selections/selections_0lep.pdf}\label{fig:multibin_scheme_0l}}
	\subfigure[]{\includegraphics[width=0.49\linewidth]{figures/strong_prod/paper/selections/selections_1lep.pdf}\label{fig:multibin_scheme_1l}}
	\caption{Scheme of the multi-bin analysis for the \subref{fig:multibin_scheme_0l} 0-lepton 
	and \subref{fig:multibin_scheme_1l} 1-lepton regions. 
        The 0L-ISR region is represented with the broad red 
	dashed line in \subref{fig:multibin_scheme_0l}. 
      }
	\label{fig:multibin_scheme}
\end{figure}

\subsection*{Control regions and validation regions}

As in the case of the cut-and-count strategy, all the \glspl{cr} require at least one signal lepton. 
In the multi-bin strategy corresponding 0L and 1L \glspl{sr} share the same \gls{cr};
This choice is motivated by the observation that 0L and 1L \glspl{sr} in the multi-bin approach are 
closer in phase space than Gtt-0L and Gtt-1L regions targeting similar signal kinematic, 
and also by the need to have \glspl{cr} that are either mutually exclusive or completely overlapping to 
allow the simultaneous inclusion of all the regions in the fit. 




\clearpage

\begin{landscape}
	\begin{table}[t]
   		\centering
        		\renewcommand{\arraystretch}{1.5}
        		\begin{tabular}{c c c c c c c c c c}
        			\toprule
			\multicolumn{10}{c}{\textbf{ High-$\njet$ regions}}\\
			\multicolumn{10}{c}{Criteria common to all regions: $\nbjet \geq 3$, ${\pt}^\mathrm{jet} > 30$~GeV } \\
			\midrule 
			Targeted kinematics  & Type & \nlep & $\dphimin$ & $\mt$ & \njet & $\mtb$ & $\mjsum$ & $\met$ & $\meff$  \\
			\midrule
			\multirow{5}{*}{\begin{minipage}{3cm}\centering High-\meff\ \\ (HH) \\ (Large \msplit) \end{minipage}} 
			& SR-0L 	& $= 0$  		& $>0.4$ 		& $-$ 		& $\ge 7$		& $>100 $ 			& $>200$ 	& $> 400 $ 				& $> 2500$ \\ 
			& SR-1L 	& $\ge 1$  	& $-$		& $> 150 $ 	& $\ge 6$		& $> 120$ 			& $>200$ 	& $> 500 $ 				& $> 2300$ \\ 
			& CR 	& $\ge 1$  	& $-$ 		& $< 150$ 	& $\ge 6$		& $> 60 $ 				& $>150$ 	& $> 300 $ 				& $> 2100$ \\ 
			& VR-0L 	& $= 0$  		& $>0.4$ 		& $-$ 		& $\ge 7$		& $<100$ if $\met>300$ 	& $-$ 	& $< 300 $ if $\mtb > 100$ 	& $> 2100$ \\
			& VR-1L 	& $\ge 1$  	& $-$ 		& $> 150$ 	& $\ge 6$		& $<140$ if $\meff>2300$	& $-$ 	& $< 500$  				& $> 2100$ \\
			\midrule
			\multirow{5}{*}{\begin{minipage}{3cm}\centering Intermediate-\meff\ \\ (HI) \\ (Intermediate \msplit) \end{minipage}} 
			& SR-0L 	& $= 0$  		& $>0.4$ 		& $-$ 		& $\ge 9$		& $> 140$ 			& $>150$ 	& $> 300 $ 				& $[1800, 2500]$ \\ 
			& SR-1L 	& $\ge 1$  	& $-$		& $> 150 $ 	& $\ge 8$		& $> 140$ 			& $>150$ 	& $> 300 $ 				& $[1800, 2300]$ \\ 
			& CR 	& $\ge 1$  	& $-$ 		& $< 150$ 	& $\ge 8$		& $> 60$ 				& $>150$ 	& $> 200 $ 				& $[1700, 2100]$ \\ 
			& VR-0L 	& $= 0$  		& $>0.4$ 		& $-$ 		& $\ge 9$		& $<140$ if $\met>300$ 	& $-$ 	& $< 300 $ if $\mtb > 140$ 	& $[1650, 2100]$ \\
			& VR-1L 	& $\ge 1$  	& $-$ 		& $> 150$ 	& $\ge 8$		& $<140$ if $\met>300$	& $-$ 	& $< 300 $ if $\mtb > 140$	& $[1600, 2100]$ \\
			\midrule
			\multirow{5}{*}{\begin{minipage}{3cm}\centering Low-\meff\ \\ (HL) \\ (Small \msplit) \end{minipage}} 
			& SR-0L 	& $= 0$  		& $>0.4$ 		& $-$ 		& $\ge 9$		& $> 140$ 			& $-$ 	& $> 300 $ 				& $[900, 1800]$ \\ 
			& SR-1L 	& $\ge 1$  	& $-$		& $> 150 $ 	& $\ge 8$		& $> 140$ 			& $-$ 	& $> 300 $ 				& $[900, 1800]$ \\ 
			& CR 	& $\ge 1$  	& $-$ 		& $< 150$ 	& $\ge 8$		& $> 130$ 			& $-$ 	& $> 250 $ 				& $[900, 1700]$ \\ 
			& VR-0L 	& $= 0$  		& $>0.4$ 		& $-$ 		& $\ge 9$		& $<140$				& $-$ 	& $> 300 $ 				& $[900, 1650]$ \\
			& VR-1L 	& $\ge 1$  	& $-$ 		& $> 150$ 	& $\ge 8$		& $<140$				& $-$ 	& $> 225 $			 	& $[900, 1650]$ \\
      			\bottomrule
    		\end{tabular}
    		 \caption{Definition of the high-$\njet$ SRs, CRs and VRs of the multi-bin analysis. All kinematic variables are
                          expressed in \gev\ except $\dphimin$, which is in radians.}
                        \label{tab:multibin_Hn}
 	\end{table}
\end{landscape}

\clearpage

\begin{landscape}
	\begin{table}[t]
		\small
   		\centering
        		\renewcommand{\arraystretch}{1.5}
                        \label{tab:multibin_In}
        		\begin{tabular}{c c c c c c c c c c c}
        			\toprule
			\multicolumn{11}{c}{\textbf{ Intermediate-$\njet$ regions}}\\
			\multicolumn{11}{c}{Criteria common to all regions: $\nbjet \geq 3$, ${\pt}^\mathrm{jet} > 30$~GeV } \\
			\midrule 
			Targeted kinematics  & Type & $N_\mathrm{lepton}$ & $\dphimin$ & $\mt$ & $\njet$ & $\leadjet = b$ or $\dphilead \leq 2.9$ & $\mtb$ & $\mjsum$  & $\met$ & $\meff$  \\
			\midrule
			\multirow{5}{*}{\begin{minipage}{3cm}\centering Intermediate-\meff\ \\ (II) \\ (Intermediate \msplit) \end{minipage}} 
			& SR-0L 	& $= 0$  		& $>0.4$ 		& $-$ 		& $[7,8]$		& \cmark		& $> 140 $ 			& $>150$ 		& $> 300 $ 				& $[1600,2500]$ \\ 
			& SR-1L 	& $\ge 1$  	& $-$		& $> 150 $ 	& $[6,7]$		& $-$ 		& $> 140$ 			& $>150$ 		& $> 300 $ 				& $[1600,2300]$ \\ 
			& CR 	& $\ge 1$  	& $-$ 		& $< 150$ 	& $[6,7]$		& \cmark 		& $> 110 $ 			& $>150$ 		& $> 200 $ 				& $[1600,2100]$ \\ 
			& VR-0L 	& $= 0$  		& $>0.4$ 		& $-$ 		& $[7,8]$		& \cmark		& $<140$ 				& $-$ 		& $> 300 $ 				& $[1450,2000]$ \\
			& VR-1L 	& $\ge 1$  	& $-$ 		& $> 150$ 	& $[6,7]$		& $-$		& $<140$				& $-$ 		& $> 225 $  				& $[1450,2000]$ \\
			\midrule
			\multirow{5}{*}{\begin{minipage}{3cm}\centering Low-\meff\ \\ (IL) \\ (Low \msplit) \end{minipage}} 
			& SR-0L 	& $= 0$  		& $>0.4$ 		& $-$ 		& $[7,8]$		& \cmark		& $> 140 $ 			& $-$ 		& $> 300 $ 				& $[800,1600]$ \\ 
			& SR-1L 	& $\ge 1$  	& $-$		& $> 150 $ 	& $[6,7]$		& $-$ 		& $> 140$ 			& $-$ 		& $> 300 $ 				& $[800,1600]$ \\ 
			& CR 	& $\ge 1$  	& $-$ 		& $< 150$ 	& $[6,7]$		& \cmark 		& $> 130 $ 			& $-$ 		& $> 300 $ 				& $[800,1600]$ \\ 
			& VR-0L 	& $= 0$  		& $>0.4$ 		& $-$ 		& $[7,8]$		& \cmark		& $<140$ 				& $-$ 		& $> 300 $ 				& $[800,1450]$ \\
			& VR-1L 	& $\ge 1$  	& $-$ 		& $> 150$ 	& $[6,7]$		& $-$		& $<140$				& $-$ 		& $> 300 $  				& $[800,1450]$ \\
      			\bottomrule
    		\end{tabular}
    \caption{Definition of the intermediate-$\njet$ SRs, CRs and VRs of the multi-bin analysis. All kinematic variables are
                          expressed in \gev\ except $\dphimin$, which is in radians. The $\leadjet = b$  requirement specifies that 
                          the leading jet is $b$-tagged.}
 	\end{table}
\end{landscape}

\clearpage

\begin{landscape}
	\begin{table}[t]
   		\centering
        		\renewcommand{\arraystretch}{1.1}	
        		\begin{tabular}{c c c c c c c c c c c }
        			\toprule
			\multicolumn{11}{c}{\textbf{ Low-$\njet$ regions}}\\
			\multicolumn{11}{c}{Criteria common to all regions: $\nbjet \geq 3$, ${\pt}^\mathrm{jet} > 30$~GeV } \\
			\midrule 
			Targeted kinematics  & Type & $N_\mathrm{lepton}$ & $\dphimin$ & $\mt$ & $\njet$ & $\leadjet = b$ or $\dphilead \leq 2.9$ & $\pt^{\fourthjet}$ & $\mtb$ & $\met$ & $\meff$  \\
			\midrule
			\multirow{3}{*}{\begin{minipage}{3cm}\centering High-\meff\ \\ (LH) \\ (Large \msplit) \end{minipage}} 
			& SR 	& $= 0$  		& $>0.4$ 		& $-$ 		& $[4,6]$		& $-$		& $>90$					& $-$ 			& $> 300 $ 				& $> 2400$ \\ 
			& CR 	& $\ge 1$  	& $-$ 		& $< 150$ 	& $[4,5]$		& $-$ 	 	& -						& $-$ 			& $> 200 $ 				& $> 2100$ \\ 
			& VR 	& $= 0$  		& $>0.4$ 		& $-$ 		& $[4,6]$		& $-$ 		& $>90$ if $\met < 300$		& $-$			& $> 200 $ 				& $[2000,2400]$ \\
			\midrule
			\multirow{3}{*}{\begin{minipage}{3cm}\centering Intermediate-\meff\ \\ (LI) \\ (Intermediate \msplit) \end{minipage}} 
			& SR 	& $= 0$  		& $>0.4$ 		& $-$ 		& $[4,6]$		& \cmark		& $>90$				& $>140$ 		& $> 350 $ 				& $[1400,2400]$ \\ 
			& CR 	& $\ge 1$  	& $-$ 		& $< 150$ 	& $[4,5]$		& \cmark 	 	& $>70$				& $-$ 		& $> 300 $ 				& $[1400,2000]$ \\ 
			& VR 	& $= 0$  		& $>0.4$ 		& $-$ 		& $[4,6]$		& \cmark		& $>90$				& $<140$ 		& $> 300 $ 				& $[1250,1800]$ \\
			\midrule
			\multirow{3}{*}{\begin{minipage}{3cm}\centering Low-\meff\ \\ (LL) \\ (Low \msplit) \end{minipage}} 
			& SR 	& $= 0$  		& $>0.4$ 		& $-$ 		& $[4,6]$		& \cmark		& $>90$				& $>140$ 		& $> 350 $ 				& $[800,1400]$ \\ 
			& CR 	& $\ge 1$  	& $-$ 		& $< 150$ 	& $[4,5]$		& \cmark 	 	& $>70$				& $-$ 		& $> 300 $ 				& $[800,1400]$ \\ 
			& VR 	& $= 0$  		& $>0.4$ 		& $-$ 		& $[4,6]$		& \cmark		& $>90$				& $<140$ 		& $> 300 $ 				& $[800,1250]$ \\
      			\bottomrule
    		\end{tabular}
		
		\par\medskip
		
    		\begin{tabular}{K{1.5cm} K{1.5cm} K{1.5cm} K{1.5cm} K{1.5cm} K{1.5cm} K{1.5cm} K{1.5cm} }
        			\toprule
			\multicolumn{8}{c}{\textbf{ ISR regions}}\\
			\multicolumn{8}{c}{Criteria common to all regions: $\nbjet \geq 3$, $\dphilead > 2.9$, ${\pt}^\leadjet > 400$~\gev, ${\pt}^\mathrm{jet} > 30$~GeV, $\leadjet \neq b$} \\
			\midrule 
			Type & $N_\mathrm{lepton}$ & $\dphimin$ & $\mt$ & $\njet$ & $\mtb$ & $\met$ & $\meff$  \\
			\midrule
			SR 	& $= 0$  		& $>0.4$ 		& $-$ 		& $[4,8]$		& $>100$ 			& $> 600 $ 				& $<2200$ \\ 
			CR 	& $\ge 1$  	& $-$ 		& $< 150$ 	& $[4,7]$		& $-$ 			& $> 400 $ 				& $<2000$ \\ 
			VR 	& $= 0$  		& $>0.4$ 		& $-$ 		& $[4,8]$		& $>100$ 			& $> 250 $ 				& $<2000$ \\
      			\bottomrule
    		\end{tabular}
\caption{Definition of the low-$\njet$ and ISR SRs, CRs and VRs of the multi-bin analysis. All kinematic variables are
                          expressed in \gev\ except $\dphimin$, which is
                          in radians. The $\leadjet = b$ ($\leadjet \neq b$) requirement specifies that 
                          the leading jet is (not) $b$-tagged.}
                        \label{tab:multibin_Ln}
                        %\label{tab:multibin_ISR}	
 	\end{table}
\end{landscape}

\clearpage 

\section{Pre-fit data-MC}
\label{sec:strong:dataMC}
In this section we show the pre-fit agreement between data and the \gls{mc} simulation before the fit in the \gls{cr}
in the distribution of the same kinematic variables as in Section \ref{sec:strong:sigbkg}. 
All the plots show in this section do not include systematic uncertainties and include all the relevant \gls{mc} \glspl{sf}. 
In order to investigate a region of the phase space depleted in signal events the requirement on the number
of b-tagged jets is relaxed to $\nbjet \geq 3$. 

\section{Kinematic Reweighting}

\section{Systematic uncertainties}
\label{sec:strong:syst}

The sources of systematic uncertainties discussed in Sections \ref{sec:common_syst} and \ref{sec:common_backgrounds} are included in the analysis.
The relative size of these uncertainties after the fit in the \glspl{cr} is shown in Figures \ref{fig:syst_cutandcount} and \ref{fig:syst_multibin},
for the cut-and-count and multi-bin analyses respectively. 
In the figure, the uncertainties are grouped into:
\begin{itemize}
\item Experimental uncertainties, that contain the sum in quadrature of the detector-related uncertainties 
presented in Section \ref{sec:common_syst}. These are considered for both the background and signal \gls{mc} samples.
The ranking of the different sources of uncertainty changes from region to region; in general the dominant ones are \gls{jes} (that has a 
relative impact on the expected background between 4 and 35\% in the different \glspl{sr}), \gls{jer} (0-26\%) and the uncertainties on the 
b-tagging efficiency and mistagging rate (3-24\%).

\item Theoretical uncertainties, which are the sum in quadrature of the modelling systematics discussed in Section \ref{sec:common_backgrounds}.
In this case the dominant uncertainties are the ones on the modelling of the \ttbar background, whose impact ranges between 5 and 76\%).

\item \gls{mc} statistical uncertainty.

\item Statistical uncertainty in the \glspl{cr}, that is reflected in the uncertainty on the \ttbar scale factor and takes values between 10 and 30\%.

\end{itemize}

The total uncertainty (black line in Figure \ref{fig:syst}) takes into account correlation effects across the different systematic sources, 
and therefore is not the sum in quadrature of the individual components. 

\begin{figure}[htbp]
	\centering
	\subfigure[]{\includegraphics[width=0.85\textwidth]{figures/strong_prod/paper/Cut_and_Count.pdf}\label{fig:syst_cutandcount}}\\
	\subfigure[]{\includegraphics[width=0.85\textwidth]{figures/strong_prod/paper/Multi_bin.pdf}\label{fig:syst_multibin}}\\
	\caption{Relative systematic uncertainty in the background estimate for the \subref{fig:syst_cutandcount} cut-and-count and \subref{fig:syst_multibin} multi-bin analyses. The individual uncertainties can be correlated, such that the total background uncertainty is not necessarily their sum in quadrature.
	} 
	\label{fig:syst}
\end{figure}

\section{Results}

\begin{figure}[htbp]
	\centering
	\subfigure[]{\includegraphics[width=0.9\textwidth]{figures/strong_prod/paper/pulls/histpull_pulls_in_CRs_17_06_23_singlebin.pdf}\label{fig:pullCR_discovery}}
	\subfigure[]{\includegraphics[width=0.9\textwidth]{figures/strong_prod/paper/pulls/histpull_pulls_in_CRs_multichannel_17_06_23.pdf}\label{fig:pullCR_exclusion}}
	\caption{Pre-fit event yield in control regions and related \ttbar
          normalization factors after the background-only fit for
          \subref{fig:pullCR_discovery} 
		the cut-and-count and \subref{fig:pullCR_exclusion} the multi-bin analyses. The upper panel shows 
		the observed number of events and the predicted background yield before the fit.
		The background category \ttbar+X includes \ttbar+W/Z, \ttbar+H and \ttbar\ttbar events. All of these
                regions require at least one signal lepton, for which the
                multijet background is negligible. All uncertainties describes in Section \ref{sec:strong:syst} are included in the uncertainty band.
		The \ttbar normalisation is obtained from the fit
                and is displayed in the bottom panel. 
	} 
	\label{fig:pullCR}
\end{figure}


\begin{figure}[htbp]
	\centering
	\subfigure[]{\includegraphics[width=0.9\textwidth]{figures/strong_prod/paper/pulls/histpull_pulls_in_VRs_17_06_23_singlebin.pdf}\label{fig:pullVR_discovery}}\\
	\subfigure[]{\includegraphics[width=0.9\textwidth]{figures/strong_prod/paper/pulls/histpull_pulls_in_VRs_multichannel_17_06_23.pdf}\label{fig:pullVR_exclusion}}\\
	\caption{Results of the background-only fit extrapolated to the VRs of \subref{fig:pullVR_discovery} the cut-and-count and \subref{fig:pullVR_exclusion}
		the multi-bin analyses. The \ttbar normalisation 
		is obtained from the fit to the CRs shown in Figure~\ref{fig:pullCR}. The upper panel shows 
		the observed number of events and the predicted background yield.
		All uncertainties  defined in Section~\ref{sec:syst} are included in the 
		uncertainty band. The background category \ttbar+X includes \ttbar+W/Z, 
		\ttbar+H and \ttbar\ttbar events. The lower panel shows the pulls in 
		each VR. 
	} 
	\label{fig:pullVR}
\end{figure}


\begin{figure}[htbp]
	\centering
	\subfigure[]{\includegraphics[width=0.9\textwidth]{figures/strong_prod/paper/pulls/histpull_pulls_in_SRs_17_06_23_singlebin.pdf}\label{fig:pullSR_discovery}}\\
	\subfigure[]{\includegraphics[width=0.9\textwidth]{figures/strong_prod/paper/pulls/histpull_pulls_in_SRs_multichannel_17_06_23.pdf}\label{fig:pullSR_exclusion}}\\
	\caption{Results of the background-only fit extrapolated to the SRs for \subref{fig:pullSR_discovery}
	the cut-and-count and \subref{fig:pullSR_exclusion} the multi-bin analyses. The data in the  SRs are 
	not included in the fit.  The upper panel shows the observed number of events and the predicted background 
	yield. All uncertainties  defined in Section~\ref{sec:syst} are included in the uncertainty band. The background 
	category $\ttbar+X$ includes $\ttbar W/Z$, $\ttbar H$ and $\ttbar \ttbar$ events. The lower panel shows the 
	pulls in each SR.} 
	\label{fig:pullSR}
\end{figure}

\clearpage 

\section{Interpretation}

\subsection{Model-independent limits}

\begin{table}[t]
        \centering
        \small
        \begin{tabular*}{0.6\textwidth}{@{\extracolsep{\fill}}lcccc}
                \noalign{\smallskip}\toprule\noalign{\smallskip}
                Signal channel         & $p_0$ (Z)            & $\sigma^{95}_\mathrm{vis}$ [fb]  &  $S_{\textrm obs}^{95}$  & $S_{\textrm exp}^{95}$   \\
                \noalign{\smallskip}\midrule \noalign{\smallskip}
                SR-Gtt-1L-B & $ 0.50~(0.00) $ &  $0.08$ &  $3.0$ & $ { 3.0 }^{ +1.0 }_{ -0.0 }$ \\[1mm]
                SR-Gtt-1L-M & $ 0.34~(0.42)$ &  $0.11$ &  $3.9$ & $ { 3.6 }^{ +1.1 }_{ -0.4 }$ \\[1mm]
                SR-Gtt-1L-C & $ 0.50~(0.00)$ &  $0.13$ &  $4.8$ & $ { 4.7 }^{ +1.8 }_{ -0.9 }$ \\[1mm]
                \noalign{\smallskip}\midrule \noalign{\smallskip}
                SR-Gtt-0L-B & $ 0.32~(0.48)$ & $0.13$ &  $4.8$ & $ { 4.1 }^{ +1.7 }_{ -0.6 }$  \\[1mm]
                SR-Gtt-0L-M & $ 0.25~(0.69)$ &  $0.21$ &  $7.5$ & $ { 6.0 }^{ +2.3 }_{ -1.4 }$ \\[1mm]
                SR-Gtt-0L-C & $ 0.50~(0.00)$ &  $0.39$ &  $14.0$ & $ { 17.8 }^{ +6.6 }_{ -4.5 }$ \\[1mm] %%to be updated
                \noalign{\smallskip}\midrule\noalign{\smallskip}
                SR-Gbb-B & $ 0.50~(0.00) $ &  $0.13$ &  $4.6$ & $ { 4.6 }^{ +1.7 }_{ -1.0 }$  \\[1mm]
                SR-Gbb-M & $ 0.50~(0.00) $ & $0.12$ &  $4.4$ & $ { 5.0 }^{ +1.9 }_{ -1.1 }$ \\[1mm]
                SR-Gbb-C & $ 0.50~(0.00) $ &  $0.18$ &  $6.6$ & $ { 6.9 }^{ +2.7 }_{ -1.8 }$ \\[1mm]
                SR-Gbb-VC & $ 0.50~(0.00) $ &  $0.08$ &  $3.0$ & $ { 4.6 }^{ +2.0 }_{ -1.3 }$\\
                \noalign{\smallskip}\midrule\noalign{\smallskip}
        \end{tabular*}
                \caption{The $p_0$-values and $Z$ (the number of equivalent Gaussian standard deviations), 
        	the 95$\%$ CL upper limits on the visible cross-section
                ($\sigma^{95}_\mathrm{vis}$),
                and the observed and
                expected 95$\%$ CL upper limits on the number of BSM events ($S_{\textrm
                obs}^{95}$ and $S_{\textrm exp}^{95}$). The maximum
              allowed $p_0$-value
              is truncated at 0.5.}
        \label{mod-ind-lim}
\end{table}

\subsection{Model-dependent limits}

\begin{figure}[htbp]
	\centering 
	\subfigure[]{\includegraphics[width=0.75\textwidth]{figures/strong_prod/paper/limits/Limits_Gtt.pdf}\label{fig:limits_Gtt}}
	\subfigure[]{\includegraphics[width=0.75\textwidth]{figures/strong_prod/paper/limits/Limits_Gbb.pdf}\label{fig:limits_Gbb}}
	\caption{Exclusion limits in the $\ninoone$ and $\gluino$ mass plane
  		for the \subref{fig:limits_Gtt} Gtt and  \subref{fig:limits_Gbb} Gbb models obtained
		in the context of the multi-bin analysis. The dashed and solid bold lines
		show the 95\% CL expected and observed limits, respectively. The
  		shaded bands around the expected limits show the
                impact of the
  		experimental and background uncertainties. The dotted
  		lines show the impact on the observed limit of the variation of the
  		nominal signal cross-section by $\pm 1 \sigma$ of its theoretical
  		uncertainty. 
		The 95\%~CL expected and observed limits from the ATLAS search based on 2015 data 
  		\cite{Aad:2016eki} are also shown.}
	\label{fig:limits_GbbGtt}
\end{figure}


\begin{figure}[htbp]
	\centering
	\subfigure[]{\includegraphics[width=0.63\textwidth]{figures/strong_prod/paper/limits/triangle_UL_massless_neutralino.pdf}\label{fig:limit_br_fixed_neu}}
	\subfigure[]{\includegraphics[width=0.63\textwidth]{figures/strong_prod/paper/limits/triangle_UL_1900_gluino.pdf}\label{fig:limit_br_fixed_glu}}
	\caption{Exclusion limits in the $\gluino \to t \bar{t} \ninoone$ and $\gluino \to b \bar{b} \ninoone$
		branching ratio plane assuming \subref{fig:limit_br_fixed_neu} a neutralino mass of 1 GeV and various gluino masses 
		(1.8, 1.9 and 2.0 TeV) and \subref{fig:limit_br_fixed_glu} a gluino mass of 1.9 TeV and three neutralino masses (1, 600 and 1000 GeV). 
		In \subref{fig:limit_br_fixed_neu}, the expected limit for a gluino mass of 1.8 TeV follows the plot axes, meaning that the whole plane is 
		expected to be excluded at 95\% CL.
		The dashed and solid bold lines show the 95\% CL expected and observed limits, respectively. The hashing indicates which side of the line 
		is excluded. The upper right half of the plane is forbidden by the requirement that the sum of branching ratios does not exceed 100\%.}
\end{figure}

\section{Comparison of cut-and-count and multi-bin strategies}

\begin{figure}[htbp]
	\centering 
	\subfigure[]{\includegraphics[width=0.75\textwidth]{figures/strong_prod/extra/UL_comp_combi_multich_Gtt.pdf}\label{fig:limits_Gtt_comp}}
	\subfigure[]{\includegraphics[width=0.75\textwidth]{figures/strong_prod/extra/UL_comp_combi_multich_Gbb.pdf}\label{fig:limits_Gbb_comp}}
	\caption{Exclusion limits in the $\ninoone$ and $\gluino$ mass plane
  		for the \subref{fig:limits_Gtt_comp} Gtt and  \subref{fig:limits_Gbb_comp} Gbb models obtained
		in the context of the multi-bin analysis (pink line) and of the cut-and-count analysis (blue line). 
		The gray numbers show the relative difference in expected \gls{ul} between the multi-bin and the cut-and-count analysis: a negative number shows a lower expected \gls{ul} for the multi-bin analysis.}
	\label{fig:limits_GbbGtt_comp}
\end{figure}

\section{Results in the context of the ATLAS SUSY group}
