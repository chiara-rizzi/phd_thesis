
\section{Signal Model}

This analysis studies the simplified model shown in Figure \ref{fig:feyn}, targeting primarily the decays via Higgs boson.
These are models of \gls{ggm} \cite{Meade:2008wd,Cheung:2007es,Dine:1981gu,AlvarezGaume:1981wy,Nappi:1982hm} 
or \gls{gmsb} \cite{Dimopoulos:1996vz,Matchev:1999ft} where the \gls{lsp} is the gravitino ($\gravino$) and 
the lightest neutralino is the \gls{nlsp}.

\begin{figure}[htbp]
	\centering
	\includegraphics[width=0.35\textwidth]{figures/ewk_prod/varie/N1N1-hhGG-bbbb_Z}
	\caption{Diagram for the simplified model considered in the analysis. The primary interpretation of the analysis is the decay via Higgs bosons, but decays via varied branching ratios to $Z$ bosons are also studied. The production of the \hino\ occurs
via mass-degenerate pairs of charginos or neutralinos, which decay to the \ninoone\ and immeasurably low momentum particles.} 
	\label{fig:feyn}
\end{figure}

As discussed in Section \ref{} \gls{susy} predicts five Higgs bosons. 
Their superpartners (higgsinos) mix with the superpartners of the electroweak gauge bosons to form charginos and neutralinos.

In models where the the lightest neutralinos and charginos are dominated by the higgsino component, the four lightest charginos 
and neutralinos are nearly degenerate \cite{Papucci:2011wy,Barbieri:2009ev,Han:2014kaa} and ordered as: $m_{\tilde\chi^0_1}~<~m_{\tilde\chi^\pm_1}~<~m_{\tilde\chi^0_2}$.

In this case the direct production of a \ninoone\ninoone\ pair is suppressed, and the production cross section is dominated by 
\ninoone\ninotwo, \ninoone\chinoonepm, \ninotwo\chinoonepm, and \chinoonep\chinoonem production.
The \ninotwo and \chinoonepm then decay to the \ninoone and soft particles that can not be detected (originating from the 
decay of off-shell W and Z bosons), therefore all of these production processes give practically the same final state as 
\ninoone\ninoone\  pair production. 

Since in this chapter we consider only the case where the lightest neutralino is dominated by the higgsino component,
we will use interchangeably the notation \ninoone or \hino to indicate it.  



\subsection{Signal cross section}

%\section{Previous Limits}

