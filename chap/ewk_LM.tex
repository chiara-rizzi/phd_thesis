\section{Complementary Analysis Targeting Low Signal Masses}
\label{sec:ewk:LM}

The analysis discussed in this chapter is limited in sensitivity for low \mhino, as it is clear from Figure \ref{fig:exclusion_high}. 
This is because when \mhino approaches the Higgs mass, the decay products (Higgs boson and \gravino)
have increasingly low \pt, and a low \pt \gravino does not produce enough \met to fire the \met trigger and be selected 
in the analysis. 
To gain sensitivity also to the low-\mhino part of the mass spectrum, which is particularly interesting for Naturalness arguments, 
this analysis is complemented by a second analysis that targets low-\met events, referred to as "low-mass" analysis. 
This analysis selects events with at least four b-tagged jets, using a b-tagging \gls{op} with an efficiency of 70\%
(tighter than the 77\% used in the high-mass analysis).
The jets used to reconstruct the Higgs candidates are the four with the highest b-tagging score, and are paired minimizing the quantity 
$D_{hh}$, defined as:
\begin{equation}
  D_{hh} = \left|m_{2j}^\textrm{lead} - \frac{120}{110}m_{2j}^\textrm{subl}\right| \; ,
\end{equation}
where $m_{2j}^\textrm{lead}$ and $m_{2j}^\textrm{subl}$ are the masses of the Higgs boson candidates with leading and subleading \pt respectively.
This pairing choice tends to create two Higgs candidates with similar mass; 
for low Higgsino masses the b-jets originating form the decay of the Higgs bosons are less collimated, 
and this choice is therefore more effective than minimizing \dRmax. 

The main background is constituted by multijet events and a small number of \ttbar events, as opposed to the
high mass analysis, where multijet is an almost-negligible background after applying the \dphimin selection. 
The background from \ttbar events is further reduced by requiring $X_{Wt}>1.8$, defined as:

\begin{equation}
 X_{Wt} = \sqrt{\left( \frac{m_W - 80.4\,\ \gev}{0.1 \times m_W} \right)^2 + \left( \frac{m_t - 172.5\,\ \gev}{0.1 \times m_t} \right)^2 } \;,
\label{eqn:xwt}
\end{equation}

\noindent where the top and W-boson candidates are built as described in Ref \cite{Aaboud:2018htj}. 
A low value of $X_{Wt}$ corresponds to a high probability of the event to be a \ttbar event. 

The \gls{sr} is defined by requiring: 
\begin{equation}
X_{hh}^\textrm{SR} = \sqrt{ \left( \frac{m_{2j}^\textrm{lead} - 120\ \gev}{0.1\times m_{2j}^\textrm{lead}} \right)^2 + \left( \frac{m_{2j}^\textrm{subl} - 110\ \gev}{0.1\times m_{2j}^\textrm{subl}} \right)^2} \ <\ 1.6,
\end{equation}
\noindent where $0.1 \times m_{2j}^\textrm{lead}$ and $0.1 \times m_{2j}^\textrm{subl}$ approximate the mass resolution of the two Higgs 
boson candidates. 

The events in the \gls{sr} are further binned based on the two-dimensional distribution of \met and \meff, 
and this is used as input in the statistical analysis. The binning used is:
\begin{eqnarray*} 
\met &=& \{0, 20, 45, 70, 100, 150, 200\} \;, \\
\meffb &=&\{160, 200, 260, 340, 440, 560, 700, 860\} \;,
\end{eqnarray*}

\noindent where the values are expressed in GeV.

\section{Combined Results}

\begin{figure}[htbp]
	\centering
\includegraphics[width=0.75\textwidth]{figures/ewk_prod/interpretation/GGMupperLimit_unblinded_jump}
	\caption{The observed (solid) vs expected (dashed) 95\% upper limits on the \hino\ pair production cross-section as a function of \mhino.  The 1$\sigma$ and 2$\sigma$ uncertainty bands on the expected limit are shown as green and yellow, respectively. The theory cross-section and its uncertainty are shown in the solid and shaded red curve.
   The results of the low-mass analysis are used below $\mhino = 300$ GeV, while those of the high-mass analysis are used above. 
   Figure from Ref \cite{Aaboud:2018htj}. } 
	\label{fig:exclusion}
\end{figure}


\begin{figure}[htbp]    
	\centering    
    \includegraphics[width=0.75\textwidth]{figures/ewk_prod/interpretation/my_br_plot_unblind_yellow_band}\label{fig:exclusion_br}
	\caption{The observed (solid) vs expected (dashed) 95\% limits in the \mhino\ vs $B(\hino\rightarrow h \tilde{G})$ plane, where $B(\hino\rightarrow h \tilde{G})$ denotes the branching ratio for the decay $\hino \rightarrow h \gravino$. The 1$\sigma$ uncertainty band is overlaid in green and the 2$\sigma$ in yellow.
	The results of the low-mass analysis are used below $\mhino = 300$ GeV, while those of the high-mass analysis are used above.
	 The regions above the lines are excluded by the analyses. Figure from Ref \cite{Aaboud:2018htj}. } 
	\label{fig:exclusion}
\end{figure}


