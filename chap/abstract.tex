\chapter*{Abstract}

This dissertation presents two searches for Supersymmetry at the Large Hadron Collider (LHC), targeting signal models with multiple top and bottom quarks in the final state.

The first search targets gluino pair production where each gluino decays through a stop
(Gtt model) or a sbottom (Gbb model) into the Lightest Sypersymmetric Partner (LSP) and into four top or bottom quarks respectively;
in both cases the final state has a high multiplicity of $b$-jets, collimated 
sprays of particles originating from the hadronization of $b$-quarks, and 
missing transverse momentum (\met) originating from the LSP  
that escapes the detection. 

The second search targets a GGM model of Higgsino pair production, 
where each Higgsino decays into a Standard Model Higgs boson and 
a gravitino, which in this case plays the role of the LSP. 
This search focuses on the decay of the two Higgs bosons into pairs of $b$-quarks, yielding again final states with multiple $b$-jets.
This is the first ATLAS analysis targeting this signature, which had been 
previously considered only in searches performed by the CMS Collaboration.

Both searches in the thesis use the data collected by the ATLAS experiment at the LHC 
between 2015 and 2016, at a center-of-mass energy  \cmtre TeV,
corresponding to an integrated luminosity of 36.1 \ifb.
The gluino search, without further reoptimization, is also extended using the data collected in 2017, 
for a total integrated luminosity of 79.8 \ifb.

No significant excess is observed in any of the search regions, 
and the results are used to 
set limits on the production of supersymmetric particles. 
The gluino analysis excludes at 95\% confidence level gluino masses up to 2.25 TeV for the Gtt model 
and 2.17 TeV for the Gbb model, for neutralino masses below 800 GeV.
The Higgsino analysis excludes Higgsino masses in the range 240-880 GeV, assuming 
that the Higgsino decays into a Higgs boson and a gravitino with 100\% branching ratio. 


\textbf{keywords}: particle physics, CERN, LHC, ATLAS, top quark, Higgs boson, new phenomena, search, Monte Carlo