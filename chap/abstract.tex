\chapter*{Abstract}

This dissertation presents two searches for Supersymmetry in proton-proton collisions at CERN's Large Hadron Collider (LHC), 
targeting signal models that lead to the production of multiple top quarks or bottom quarks in the final state.

The first search targets gluino pair production, where each gluino decays through a top squark (Gtt model) or a bottom squark (Gbb model) 
into a top-antitop quark pair or a bottom-antibottom quark pair, respectively, and a neutralino, which is the Lightest Sypersymmetric Partner (LSP).
Each top quark in turn decays into a $W$ boson and an a bottom quark.
Thus, the final state is characterized by a high multiplicity of bottom jets, which are collimated sprays of particles originating from the hadronization of bottom quarks, and 
missing transverse momentum (\met) from the LSP that escapes the detection. 

The second search targets a GGM model of Higgsino pair production, 
where each Higgsino decays into a Standard Model Higgs boson and 
a gravitino, which in this case plays the role of the LSP. 
This search focuses on the decay of the two Higgs bosons into bottom-antibottom quark pairs, yielding again final states with multiple $b$-jets.
This is the first ATLAS analysis targeting this signature, which had been 
previously considered in searches performed by the CMS Collaboration.

Both searches in the thesis use the data collected by the ATLAS experiment at the LHC 
between 2015 and 2016, at a center-of-mass energy  \cmtre TeV,
corresponding to an integrated luminosity of 36.1 \ifb.
The gluino search, without further reoptimization, is also extended using the data collected in 2017, for a total integrated luminosity of 79.8 \ifb.

No significant excess of events above the Standard Model expectation is observed in any of the search regions, 
and the results are used to set upper limits on the production of supersymmetric particles. 
The first search excludes at 95\% confidence level gluino masses up to 2.25 TeV for the Gtt model 
and up to 2.17 TeV for the Gbb model, in both cases for neutralino masses below 800 GeV.
The second search excludes Higgsino masses in the range 240--880 GeV, assuming 
that the Higgsino decays exclusively into a Higgs boson and a gravitino. 

\par\bigskip
\par\bigskip 
\par\bigskip

\noindent \textbf{Keywords}: particle physics, CERN, LHC, ATLAS, supersymmetry, new phenomena, search, top quark, Higgs boson. 