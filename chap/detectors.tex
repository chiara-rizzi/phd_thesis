\section{Detectors for collider Physics}
\label{sec:detectors}

In this section we review the basic concepts that drive the design of the \gls{lhc} detectors, including \gls{atlas}.

\subsection{Identification of particles}
\label{sec:detectors:identification}



The ability to accurately identify particles and reconstruct their energy and trajectory is what drives the design of detectors for high energy physics. In a detector, different sub-systems are able to capture different types of particle interactions, and the combination of the information collected by each of them allows to identify particles (or at least assign them to families, such as neutral or charged hadrons). A typical schema of the subdetectors sequence is shown in Fig. \ref{fig:detector:interaction}. The innermost layer, closer to the interaction point, is the \textit{tracking system}, dedicated to the measurement of the signed charge and momentum of charged particles. The following layers are the electromagnetic and hadronic \textit{calorimeters}, that measure the energy of particles with electromagnetic and hadronic interactions respectively. The outermost layer is dedicated to the \textit{muon system}: because of their large mass (about 200 times more than electrons) muons do not produce electromagnetic showers and are therefore easy to identify as they are the only detectable particles that reach the external part of the detector.

\begin{figure}[ht]
\centering
\includegraphics[width=0.6\textwidth]{figures/detector/particles_in_detector}
\caption{Components of a typical detector for physics at colliders. Different particles are identified by the distinctive signatures in the subdetectors. Figure from Ref. \cite{Lippmann:2011bb}.}
\label{fig:detector:interaction}
\end{figure}


\subsection{Tracking and spectrometry}
\label{sec:dec:tracking}
A tracking device measures the traces left by charged particles passing through it. To allow the determination of the momentum and the charge of a particle, a tracking device needs to be accompanied by a magnetic field (in this case we speak of a \textit{magnetic spectrometer}): once the magnetic field is known, the measure of the radius of curvature of the particle is equivalent to a measure of its momentum, according to Eq. \ref{eq:cern:p03br}. In a typical particle detector as the one in the schema in Fig. \ref{fig:detector:interaction}, both the inner tracking system and the muon system are magnetic spectrometers.

The relative uncertainty on the momentum is given by:
\begin{equation}
\frac{\sigma_p}{p} = \sqrt{ \left(a p \right)^2 + b^2} \; .
\label{eq:tracking:reso}
\end{equation}

The first term, whose relative importance increases for high-momentum particles, derives from the resolution in the measurement of the curvature. Typical values for $a$ are between 0.01\% and 1\%. The constant term in Eq. \ref{eq:tracking:reso} accounts for the impact of multiple Coulomb scattering, which broadens the distribution of the transverse momentum perpendicular to the direction of motion. This terms is important only at low energies, while it is negligible for high-energy particles.

There are three main configurations of magnetic fields typically used in momentum spectrometers:
\begin{itemize}
\item A \textit{dipole} field leads to a rectangular symmetry; if we think of a circular collider with a coordinate system where $z$ is the direction along the beam trajectory and ($x$,$y$) define a Cartesian system in the transverse plane, a dipole  field in the $x$ direction will cause a deflection in the ($y,z$) plane. This is the configuration adopted by forward spectrometers like LHCb, where the tracking devices are arranged in sequence in the $z$ direction. As an example, the integral of the LHCb dipole field over the detector length is 4 Tm.

\item A \textit{solenoidal} field leads to a cylindrical symmetry and, if the field lines are along the $z$ direction, the deflection is in the ($x,y$) plane. This is the typical configuration of the spectrometers in the central barrel, where the detectors are arranged in cylindrical layers. The CMS solenoid field is 4 T, while the \gls{atlas} one is 2 T. 

\item A \textit{toroidal} field leads to an azymuthal symmetry: the direction of the field lines is a circle in the transverse plane, and the deflection is in the ($r,z$) plane. This configuration is adopted by the \gls{atlas} muon spectrometer, with a magnetic field of 4 T. As in the case of the solenoidal field, the detectors are arranged in cylindrical shells.
\end{itemize}

The track left by a charged particle curving because of the magnetic field is reconstructed by the tracking detectors. The two most common categories of tracking devices are gas and silicon detectors, described in the next two sections.

\subsubsection*{Gas detectors}

In gas detectors, the passage of a charged particle ionizes the atoms and creates \textit{electron-ion pairs}. Once the electron and the ion are created, they can be separated (thus creating a current) by applying an electric field. This induces signal on an electrode added to the material, read through a readout system. The basic principle of most of gas detectors relies on a tube with a wire in the center, which is an anode with a high electric field. When the particle crosses the gas, the ionization electrons drift toward the anode and can be collected. 
Without an electric field, the electron and ion would move in the system by thermal diffusion. The effect of an electric field is to make the electron and ion move in opposite directions, allowing us to measure them. Since not many electrons are produced in gas, they need to be amplified. Inside the tube, the electric field decreases with the inverse of the distance from the anode wire. When the electrons reach a distance of a few micrometers from the wire, the electric field is very large and the electrons gain more energy than the ionization energy; this leads to secondary ionization and an exponential growth of the number of electron-ion pairs (\textit{avalanche effect}). 

Several types of gas detectors with different characteristics have been developed. In \textit{single-wire proportional chambers}, like the one used in the outer tracker of the LHCb experiment, several counters are combined next to each other, to allow the measurement of the particle position. In \textit{multi-wire proportional chambers} \cite{CHARPAK1968262} several wires in two perpendicular directions are contained in the same box filled with gas. 
\textit{Drift chambers} are a further evolution of wire chambers, where the position of the particle is computed by measuring the time taken by the 
secondary ionization to reach the anode. \textit{Time projection chambers} (TPC) are a different type of gas detector, used for example in ALICE. 
It is constituted by a large area filled only with gas, without wires, with detectors and readout structure only at the end plates. 
Between the plates a strong electric field causes the electron-ion pairs to drift, 
until they reach the plates where a combined measurement of the position and time of arrival allows to reconstruct the three-dimensional 
trajectory of the particle (that can be curved if a magnetic field is added). 
\textit{Micro-strips chambers} \cite{OED1988351} are more modern, very condensed and thin gas detectors. 
Instead of being generated by a wire, the electric field comes from small metal deposits on a high-resistivity substrate.   

In general the main problems of gas detectors based on ionization are the spatial extension and the long drift time, 
which nevertheless make them suitable for the outer part of the \gls{lhc} detectors, in particular muon spectrometers.

All the gas detectors described above are based on the electron-ion pairs created by the ionization induced by the charged particle. A different principle is used in \gls{trd}. \glspl{trd} are based on detecting the electromagnetic radiation emitted by particles that cross boundaries between different media below the Cherenkov threshold \cite{1402-4896-1982-T2A-024}. The energy radiated increases with the energy of the incoming particle. These detectors are less common for tracking, but are mentioned here as the main example of a modern \gls{trd} in high-energy physics is the \gls{trt} in the inner detector of the \gls{atlas} experiment.

\subsubsection*{Solid-state detectors}

Semiconductor (and in particular silicon) detectors are the main type of solid-state detectors, and are currently the most used for inner tracking, 
where high precision and low occupancy are needed \cite{Hartmann:2009zza}. 
These detectors detectors are also based on the ionization of the material but in this case, since the structure is a crystal, 
we talk about electrons and \textit{electron holes}. 

The underlying principle is based on the \textit{band model} of solids, describing the allowed energy levels for electrons in a solid: when many atoms of the same type are bound together in a crystal lattice, in order to fulfill Pauli's principle the atomic orbitals split into many closely spaced molecular orbitals, that can be considered as continuous energy bands. The highest-energy full band is the \textit{valence band}, while the lowest partially-filled (or empty) band is the \textit{conduction band}; the energy difference between the conduction band and the valence band is referred to as \textit{energy gap}, as illustrated in Fig. \ref{fig:det:band}(a). 

\begin{figure}[ht]
\centering
\subfigure[]{\includegraphics[width=0.45\textwidth]{figures/detector/band_structure}\label{fig:det:band:a}}
\subfigure[]{\includegraphics[width=0.45\textwidth]{figures/detector/pn.pdf}\label{fig:det:band:b}}
\caption{\subref{fig:det:band:a} Schema of band model of solids. \subref{fig:det:band:b} Top: schema of a p-n junction. Bottom: charge distribution in a p-n junction and extension of the depletion area. Figures from Ref. \cite{grupen_shwartz_2008}.}
\label{fig:det:band}
\end{figure}

Semiconductors are materials where the conduction band is almost empty, but the energy gap is small ($\approx$ 1 eV, e.g. 1.07 eV for silicon), so the conduction band can be occupied by excited electrons from the valence band; this leaves holes in the valence band, that under the effect of an electric field can drift as well. Semiconductors with a pure composition have the same amount of electorns and holes. If the semiconductor is doped with an atom with one more electron in the valence band, this creates an excess of electrons and the material is referred to as a \textit{n-type} semiconductor. If instead the impurities are electron acceptors, the material has an excess of holes and is referred to as \textit{p-type} semiconductor. When a p-type and a n-type semiconductors are pulled together, the holes of the p-side drift into the n-side, and the electrons from the n-side into the p-side, as shown in Fig. \ref{fig:det:band}(b). This creates an electric field that takes charge carriers out of the area where it is present (\textit{depletion region}). When a charged particle passes through the semiconductor, it produces electron-hole pairs in the depletion region; these drift apart because of the electric field and can be collected by electrodes. The application of a negative potential difference between the p-side and the n-side can increase the depletion area.

The density of silicon is about one thousand times higher than that of gases like argon, and silicon also needs a much lower energy to be ionized (3.6 eV for silicon, while it is about 250 times higher for argon). This leads to a much higher number of electron-holes pairs than the number of electron-ion pairs in gas detectors, so the signal needs very little amplification, and the size of the detector itself can be smaller. 

Different configurations of the silicon sensors are possible. Fig. \ref{fig:det:silicon_schema}(a) shows the schema of a \textit{single-sided strip} sensor, where the readout strips are at negative potential. The second coordinate can be determined if also the n-side is divided into strips in a direction orthogonal to the ones in the p-side, as shown in Fig. \ref{fig:det:silicon_schema}(b). The grid structure of this \textit{double-sided strip} sensors can still lead to ambiguities when the number of hits is elevated; a true two-dimensional sensitivity is offered only by the \textit{pixel} modules, schematically shown in Fig. \ref{fig:det:silicon_schema}(c), where the module is divided in a matrix-like shape.

\begin{figure}[ht]
\centering
\subfigure[]{\includegraphics[width=0.32\textwidth]{figures/detector/single-sided-AC-coupled-silicon-strip.pdf}\label{fig:det:silicon_schema:a}}
\subfigure[]{\includegraphics[width=0.32\textwidth]{figures/detector/double-sided-silicon-strip.pdf}\label{fig:det:silicon_schema:b}}
\subfigure[]{\includegraphics[width=0.32\textwidth]{figures/detector/pixel-schema.pdf}\label{fig:det:silicon_schema:c}}
\caption{Schematic view of \subref{fig:det:silicon_schema:a} a single-sided strip sensor,
\subref{fig:det:silicon_schema:b} a double-sided strip sensor and \subref{fig:det:silicon_schema:c} a pixel module. }
\label{fig:det:silicon_schema}
\end{figure}



\subsection{Calorimetry}
\label{sec:dec:calo}

Calorimeters can determine the energy of both charged and neutral particles through a destructive measurement: the energy of the particles is deposited in the detector material and transformed into a measurable quantity. Because of their sensitivity to a wide variety of particles, good energy resolution and relatively small size, they are very attractive devices for accelerator physics experiments \cite{RevModPhys.75.1243,Wigmans:2000vf}.

When the incident particle interacts with the material of the calorimeter it develops a cascade of particles (\textit{shower}), with different characteristics for electromagnetic and hadronic interactions, described in the next two sections. Different types of calorimeters are necessary to capture the two typologies. The energy of the shower is decreased by the interactions happening in the \textit{absorber} material, while the \textit{active} material provides the conversion of the energy into a charge or light signal. In \textit{sampling calorimeters} layers of absorber and active material are alternated in sequence, while in \textit{homogeneous calorimeters} a single material carries out both functions.


\subsubsection*{Electromagnetic calorimeters}

The type of interaction that electromagnetically interacting particles have with the detector depends on their energy. The average fractional energy loss in lead for electrons and positrons and the photon interaction cross section in lead are shown in Fig. \ref{fig:det:xsec_elec} (a) and (b) respectively. 

\begin{figure}[ht]
\centering
\subfigure[]{\includegraphics[width=0.54\textwidth]{figures/detector/electron_energy_loss}}
\subfigure[]{\includegraphics[width=0.36\textwidth]{figures/detector/photon_xsec}}
\caption{(a) Fractional energy loss per radiation length in lead as a function of the electron (positron) energy. (b) Photon interaction cross section in lead. Figures from Ref. \cite{Patrignani:2016xqp}. }
\label{fig:det:xsec_elec}
\end{figure}


When electrons, positrons and photons with energies above 1 GeV traverse a block of material they produce a cascade of particles (\textit{electromagnetic shower}): electrons and positrons can emit a photon by Bremsstrahlung, and a photon, thanks to the interaction with a nucleus, can turn into an electron-positron pair (pair production is indicated with the symbol $\kappa_{nucl}$ in Fig. \ref{fig:det:xsec_elec} (b)). A schematic view of the evolution of an electromagnetic shower is shown  in Fig. \ref{fig:det:shower_elec}(a). 

\begin{figure}[ht]
\centering
\subfigure[]{\includegraphics[width=0.45\textwidth]{figures/detector/elec_shower} \label{fig:det:shower_elec:a}}
\subfigure[]{\includegraphics[width=0.52\textwidth]{figures/detector/elec_shower_lateral}\label{fig:det:shower_elec:b}}
\caption{\subref{fig:det:shower_elec:a} Sketch of the evolution of an electromagnetic shower. 
\subref{fig:det:shower_elec:b} Lateral and longitudinal evolution of the shower from electrons with an energy of 6 GeV. Figure from Ref. \cite{grupen_shwartz_2008}.}
\label{fig:det:shower_elec}
\end{figure}


The main parameter to describe the evolution of an electromagnetic shower is the \textit{radiation length} ($X_0$), defined as the distance over which an electron reduces its energy to $\frac{1}{e}$ of the initial value, and it corresponds also to $\frac{7}{9}$ of the mean free path for pair production for a photon. The radiation length depends on the characteristics of the material:
\begin{equation}
X_0 [\frac{g}{cm^2}] = \frac{716 \frac{g}{ cm^2} A }{Z(Z+1) \ln\left(287/\sqrt{Z}\right)} \; ,
\end{equation}

\noindent where $A$ and $Z$ are the atomic and mass number of the material. If we define $t = \frac{x}{X_0}$ as the shower depth relative to the radiation length, the maximum number of produced particles occurs at:
\begin{equation}
t_{max} = \frac{\ln\left(E_0/E_c\right)}{ln\left(2\right)} \;.
\end{equation}
Typical values for the interaction length are of the order of the cm (e.g. 0.56 cm for lead, 1.76 cm for iron \cite{Patrignani:2016xqp}); 99\% of the shower is contained in about 11(22) $X_0$ for a particle with an energy of 1 GeV(TeV), 
allowing for electromagnetic calorimeters of compact dimensions. 
The lateral with of the shower, determined mainly by multiple scattering, increases with depth and is defined in terms of the Moli\'ere radius:
\begin{equation}
R_M = \frac{21 MeV \; X_0[\frac{g}{cm^2}]}{E_c [MeV]} \; .
\end{equation}
A cylinder of radius 2$R_M$ contains about 95\% of the shower; for most calorimeters $R_M$ has a value of few centimeters, so electromagnetic showers are quite narrow. The longitudinal and lateral development of the shower induced in lead by electrons with an energy of 6 GeV is shown in Fig. \ref{fig:det:shower_elec}(b).

Once the electrons in the shower have an energy lower than the \textit{critical energy} 
($E_c$, defined as the energy where the loss through Bremsstrahlung equals the loss through ionization), 
the shower stops as the energy is dissipated mostly through ionization for electrons and photoelectric effect for photons, 
and no longer through the creation of new particles. 
Therefore all the energy of the incoming particle is in the end used to ionize the material of the detector, and this is the effect that is detected.


We have discussed in Section \ref{sec:dec:tracking} how the resolution of the momentum measurement in a magnetic spectrometer decreases with the increase in the momentum itself. Instead, the relative energy resolution in a calorimeter improves for high-energy particles, and can be written in the parametric form:

\begin{equation}
\frac{\sigma_E}{E} = \sqrt{\left(\frac{a}{\sqrt{E}} \right)^2 + \left( \frac{b}{E} \right)^2 + c^2 } \; .
\end{equation}

\noindent The first term of the sum in quadrature reflects the \textit{stochastic} nature of the shower development: ignoring the instrumental effects, the energy resolution of a calorimeter is proportional to the square root of the total track length, which is in turn proportional to the initial energy. The contribution of this term is small in homogeneous calorimeters, while is larger in sampling calorimeters (because of fluctuations of the fraction of energy deposited in the absorber) and it grows with the thickness of the absorber layers; typical values for $a$ are 5-20\% if the energy is expressed in GeV. The second term is the \textit{noise term} coming from the electronic noise of the readout chain; this term is in general more relevant for calorimeters producing charge signals than for those producing light signals, and can become the dominant term for particles with energy below one GeV. The last term is a constant deriving from instrumental effects that produce a non-uniform detector response, including for example energy lost outside the detector volume and radiation damage; this becomes the dominant term at high energies and is typically $<1\%$. 



\subsubsection*{Hadronic calorimeters}

The difference between electromagnetic calorimeters and hadronic calorimeters finds its origin in the more complicated nature of strong interactions compared to the electromagnetic ones. 

A sketch of the evolution of an hadronic shower is shown in Fig. \ref{fig:det:shower_had}(a). A first relevant difference between hadronic and electromagnetic showers is that the former has a much larger spatial extension. On the longitudinal direction, the scale is determined by the \textit{nuclear interaction length} ($\lambda_I$), which is material-dependent and can be expressed as:
\begin{equation}
\lambda_I = 35 \frac{g}{cm^2} A^{1/3} \; ,
\end{equation}

where $A$ is the atomic number of the material. For most materials used in particle detectors this results to be larger than $X_0$ (e.g. the nuclear interaction length is 17.59 cm for lead, 16.77 cm for iron \cite{Patrignani:2016xqp}). The 99\% shower containment is reached after about 5(9) $\lambda_I$ for pions of 10(138) GeV. Also the lateral width of the sower results larger than in electromagnetic interactions: while the size of the Moli\'ere radius is determined mainly by multiple scattering, the lateral profile of an hadronic shower depends on the transverse-momentum transfer, that can be quite sizable in strong nuclear interactions. Fig. \ref{fig:det:shower_had}(b) shows the lateral shower profile for 10-GeV pions in iron.

\begin{figure}[ht]
\centering
\subfigure[]{\includegraphics[width=0.48\textwidth]{figures/detector/hadron_shower}}
\subfigure[]{\includegraphics[width=0.48\textwidth]{figures/detector/hadron_shower_lateral}}
\caption{(a) Sketch of the evolution of an hadronic shower. Figure from Ref. \cite{grupen_shwartz_2008}. (b) Lateral energy distribution of shower induced by 10-GeV $\pi^-$, measured at a depth of 10, 20, 30, 50 and 70 cm in Fe. Figure from Ref. \cite{FRIEND1976505}.}
\label{fig:det:shower_had}
\end{figure}


Another difference with electromagnetic showers lies in the composition of the shower: while an electromagnetic shower is constituted only by electrons, positrons and photons, a much larger variety of particles participates to hadronic showers, including both hadrons and electromagnetically-interacting particles. Starting from the simplifying assumption that one third of the particles produced in nuclear interactions are neutral pions ($f_{\pi^0}=\frac{1}{3}$), a first approximation of the electromagnetic fraction of a shower is given by:
\begin{equation}
f_{em} = 1 - \left(1 - \frac{1}{3} \right)^n \; ,
\end{equation}
where $n$ is the number of generations in the shower. Since the number of generations increases with the initial energy, it is intuitive that also $f_{em}$ will be larger for particles of higher energy. It is found \cite{GABRIEL1994336}:
\begin{equation}
f_{em} = 1 - \left(\frac{E}{E_0}\right)^{\left( k-1 \right)} \; ,
\end{equation}
where $E_0$ is the energy necessary to produce one pion (e.g. 0.7 GeV for iron and 1.3 GeV for lead), and $k$ is a slope parameter related to $f_{\pi^0}$ through the average multiplicity $<m>$:
\begin{equation}
1-f_{\pi^0} = <m>^{\left( k-1 \right)} \;.
\end{equation}


\begin{figure}[ht]
\centering
\subfigure{\includegraphics[width=0.42\textwidth]{figures/detector/had_shower_spectra}}
\caption{Particle spectra produced by 100-GeV protons absorbed by lead, as simulated by the Fluka code \cite{Ferrari:898301} and averaged over many showers.}
\label{fig:det:shower_had_spectra}
\end{figure}

Fig. \ref{fig:det:shower_had_spectra} shows the particle spectra produced by 100-GeV protons absorbed by lead, as simulated by the Fluka code \cite{Ferrari:898301}. We can see that at low energies the particle content is dominated by electrons, positrons, photons and neutrons.

While in electromagnetic showers most of the initial energy is recorded in the detector, in hadronic showers a relevant fraction (up to 30-40\%) is invisible. This invisible fraction is caused by energy that goes into breaking the nuclear bonds, nuclear fragments that in sampling calorimeters do not reach the active material, and neutral particles that can escape the calorimeter (e.g. neutrinos or long-lived neutral kaons). Therefore, for the same initial energy, the visible energy will be lower for an hadronic shower than for an electromagnetic one. If we define the \textit{response} as the collected signal per unit of incident energy, the invisible energy causes a different response of calorimeters to the electromagnetic and to the purely-hadronic parts of the shower. Defining $\eta_{em}$, $\eta_{h}$, $\eta_{\pi}$ respectively as the calorimeter response to electromagnetic shower, the hadronic part of the shower and to pions, we have that:
\begin{equation}
\eta_{\pi} = f_{em}\eta_{em} + (1-f_{em}) \eta_h = \eta_h \left( \frac{\eta_{em}}{\eta_{h}}f_{em} + (1-f_{em})  \right) \; .
\end{equation}

In general $\frac{\eta_{em}}{\eta_{h}}>1$. Since the value of the electromagnetic fraction is energy-dependent, the signal from the calorimeter does not increase linearly with energy. \textit{Compensating} calorimeters are the ones aiming at having the same response to the electromagnetic and hadronic part of the shower ($\frac{\eta_{em}}{\eta_{h}}=1$), restoring the linearity of the calorimeter response. In non-compensating calorimeters the fluctuations in $f_{em}$ are the dominant component of the energy resolution and, since the fluctuations are Gaussian, give rise to a term proportional to:

\begin{equation}
\frac{\sigma_E}{E} = \frac{\mathrm{const}}{\sqrt{E}}  \; .
\end{equation}
In modern non-compensating calorimeters the value of the constant is about 0.4, while it can be as low as 0.2 in the case of compensating calorimeters.


\subsection{Detecting photons}

As already mentioned in the previous sections, photons are often the effect of the passage of a particle; this is the case for example in \glspl{trd} or in calorimeters with scintillator as active material. This sections is an overview of how these photon can generate a detectable current; for a more extensive discussion see for example Ref. \cite{lightdetection,Grupen:2012zpa}. The typical process of photon detection can be summarized in three different steps: first of all, the incident photon generates a photoelectron or an electron-hole pair through photoelectric or photoconductive effect. Then the signal of the electron needs to be amplified, and finally collected. Important properties to classify photon-detecting devices are \cite{Patrignani:2016xqp}:
\begin{description}[font=\normalfont]
\item[\textit{Quantum efficiency}:] the probability that the incident photon generates a photoelectron. This depends on the photon wavelength.
\item[\textit{Collection efficiency}:] the probability that the photoelectron is collected at the end of the chain.
\item[\textit{Photon detection efficiency}:] the product of quantum and collection efficiency.
\item[\textit{Gain}:] the amplification of the photoelectron, quantified as the number of electrons collected for each photoelectron generated.
\item[\textit{Dark current or dark noise}:] the output current in the absence of signal.
\item[\textit{Energy resolution}:] resolution depending on electronic noise and statistical fluctuations.
\item[\textit{Dynamic range}:] Maximum intensity that the detector can handle expressed in units of the smallest intensity with a signal-to-noise ratio above one.
\item[\textit{Time dependence}] Time between the arrival of the photon and the collection of the electrical current.
\item[\textit{Rate capability}] Inverse of the time needed after the arrival of a photon to be ready for another one.
\end{description}

Photon detectors can be broadly classified into three categories: vacuum, gaseous and solid-state detectors, described in the following paragraphs.

\subsubsection*{Photomultiplier tubes}  

\glspl{pmt} are the most common type of vacuum photon detector. A sketch of the structure of a \gls{pmt} is shown in Fig. \ref{fig:det:pmt}. It consists of a vacuum tube with an input window, a photocatode, a focusing electrode, electron multipliers and an anode  \cite{hamamatsu}. The photons enter the device through the window and excite the electrons in the valence band of the photocatode, which is a semiconductor; in \textit{transmission-type} \glspl{pmt} the photocatode is deposited on the inside of the window, while in \textit{reflection-type} \glspl{pmt} is on a separate surface. The excited electrons diffuse to the surface of the semiconductor and, if they have enough energy to overcame the vacuum level barrier, are emitted as photoelectrons. These are accelerated and focused by the electrode toward the multi-stage dynodes, a system of electrodes coated with a secondary emissive material, where the incident electrons are multiplied. The gain $G$ of the \gls{pmt} depends on the applied voltage $V$ as $G=AV^{kn}$, where $A$ and $k$ are constants and $n$ is the number of multiplicative stages. Typical values for the gain are $10^5-10^6$. After the last stage of multiplication, the electrons are collected by the anode that then outputs the current to the external circuit.

\begin{figure}[ht]
\centering
\subfigure{\includegraphics[width=0.72\textwidth]{figures/detector/pmt}}
\caption{Figure from Ref. \cite{hamamatsu}.}
\label{fig:det:pmt}
\end{figure}

Beside \glspl{pmt}, other examples of vacuum photon detectors are \gls{mcp} and \gls{hpd}. \glspl{mcp} are based on the same principle as \glspl{pmt} but substitute the discrete multiplicative stages of the dynodes with continuous multiplication in cylindrical holes of a few $\mu$m. The decreased size of a \gls{mcp} comes at the price of a large recovery time, shorter lifetime and smaller gain (the typical gain of a \gls{mcp} is about $10^4$). In \glspl{hpd} photoelectrons are accelerated onto a silicon sensor, allowing higher resolution in space and energy; this type of sensors are used for example in the CMS hadronic calorimeter.

\subsubsection*{Gaseous photon detectors}
In \textit{gaseous photon detectors} the photoelectrons are multiplied through the avalanche effect in an high-field region, similarly to what happens in gaseous tracking detectors. The photoelectrons are generated by the interaction of the photon either with a solid photocatode or with a photosensitive molecule vaporized and mixed in the gas itself. 

\subsubsection*{Solid-State photon detectors}
\textit{Solid state photon detectors} are devices where the production and detection of the photoelectrons takes place in the same semiconductor material. They are in rapid development (see e.g. Ref. \cite{Renker:2009zz}) as they provide a smaller and often cheaper option to \glspl{pmt} and gaseous detectors, especially when the area to be covered is small. Silicon photodiodes are devices used in many applications from high-energy physics to solar cells, based on a reverse-biased p-n junction. Photons passing through the silicon create electron-hole pairs through the photoconductive effect, and these electron-hole pairs are then collected respectively at the positive and negative side of the chip. 

 %Advances in solid state photon detectors


