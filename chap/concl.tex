\chapter*{Conclusion}
\addchaptertocentry{Conclusion} 

This document presented two searches targeting \gls{susy} signals leading to final states with 
high $b$-jet multiplicity, using the data collected by the \gls{atlas} experiment at the \gls{lhc} 
between 2015 and 2016, at a center-of-mass energy  \cmtre TeV. 
This dataset corresponds to an integrated luminosity of 36.1 \ifb.

The first analysis searches for gluino pair production where each gluino decays through a stop or a sbottom 
to respectively four top or bottom quarks and the \gls{lsp}, leading to a final state with four $b$-jets and 
missing transverse momentum (\met). 
This analysis employs two different strategies: cut-and-count, with several non-orthogonal \glspl{sr} optimized to 
maximize the discovery significance for selected benchmark models, and multi-bin, with orthogonal \glspl{sr} 
that are statistically combined in the maximum-likelihood fit 
to maximize the exclusion power of the analysis. 
In all the \glspl{sr} semi-leptonic \ttbar+jets constitutes the dominant background.
No significant excess is found in any of the analysis regions. 
The largest deviation between expected and observed number of events 
is in SR-0L-HH, one of the multi-bin \glspl{sr}, and it has a significance of approximately 
2.3 standard deviations. 




The second search targets \gls{ggm} model of Higgsino pair production, where each Higgsino then decays to a Higgs boson and 
a gravitino, which in this case is the \gls{lsp}. The search is performed in the channel with four $b$-jets, originating from 
the decay of the two Higgs bosons, and \met. 
This was the first \gls{atlas} analysis targeting this signature, that had been 
previously considered only in searches performed by the \gls{cms} collaboration.






