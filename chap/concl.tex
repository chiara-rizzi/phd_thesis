\chapter*{Conclusion}
\addchaptertocentry{Conclusion} 

This dissertation presented two searches targeting \gls{susy} signals leading to final states with 
high $b$-jet multiplicity, using the data collected by the \gls{atlas} experiment at the \gls{lhc} 
between 2015 and 2016, at a center-of-mass energy  \cmtre TeV. 
This dataset corresponds to an integrated luminosity of 36.1 \ifb.

The first analysis searches for gluino pair production where each gluino decays through a stop or a sbottom 
to respectively four top or bottom quarks and the \gls{lsp}, leading to a final state with four $b$-jets and 
missing transverse momentum (\met). 
This analysis employs two different strategies: cut-and-count, with several non-orthogonal \glspl{sr} optimized to 
maximize the discovery significance for selected benchmark models, and multi-bin, with orthogonal \glspl{sr} 
that are statistically combined in the maximum-likelihood fit 
to maximize the exclusion power of the analysis. 
In all the \glspl{sr} semi-leptonic \ttbar+jets constitutes the dominant background.
No significant excess is found in any of the analysis regions. 
The largest deviation between expected and observed number of events 
is in SR-0L-HH, one of the multi-bin \glspl{sr}, and it has a significance of approximately 
2.3 standard deviations. 
Exclusion limits in the m(\gluino)-m(\ninoone) are set for the two simplified models assuming 
100\% \gls{br} for the gluino into $t\bar{t}\ninoone$ and $b\bar{b}\ninoone$, denoted as "Gtt" and "Gbb" respectively.
In the case of the Gtt model, the expected and observed limit at 95\% \gls{cl} for massless neutralino 
are 2.14 and 1.97 TeV respectively; the main reason of the difference is the excess in SR-0L-HH.
Also for the Gbb model the observed limit is slightly weaker than the expected one: 
while the expected limit for massless neutralino is 2.01 TeV, the observed is 1.92 TeV.
The results of this analysis are reinterpreted also allowing a variable \gls{br} of the gluino 
into tt\ninoone, bb\ninoone and tb\chinoonepm. 
These limits, as well as the other limits discussed in this thesis, are obtained with the \gls{cls} 
prescription. These results are published in Ref. \cite{Aaboud:2017hrg}.


The mild excess in SR-0L-HH has been verified also with the 2017 dataset,
leaving the definition of the \glspl{sr} unchanged; 
together with the 2015 and 2016 data-taking periods, this leads to a total integrated luminosity 
of 79.8 \ifb. 
The update of the analysis did not find an excess in SR-0L-HH, and the increase in luminosity 
allowed to set more stringent limits on gluino pair production. 
In the case of the Gtt model, for neutralino masses below 800 GeV we excluded gluino masses up to 
2.25 TeV, while for the Gbb model the limit is at 2.17 TeV. 
The results of the analysis of the 79.8 \ifb are reinterpreted also for signals with on-shell 
stops. In this case it is found that, while for most m($\tilde{t}$) the sensitivity 
is close to the one obtained in the off-shell case, it becomes weaker 
when m($\tilde{t}$) is close to the kinematic boundary of 
m(\ninoone)+m($t$) and m(\gluino)-m($t$).  
These results have been released in Ref. \cite{ATLAS-CONF-2018-041}.


The second search presented in this thesis targets a \gls{ggm} model of higgsino pair production, 
where each higgsino then decays to a Higgs boson and 
a gravitino, which in this case is the \gls{lsp}. The search is performed in the channel with four $b$-jets, originating from 
the decay of the two Higgs bosons, and \met. 
This was the first \gls{atlas} analysis targeting this signature, that had been 
previously considered only in searches performed by the \gls{cms} collaboration.
This analysis relies on the identification of two Higgs boson candidates in events with at least three or 
at least four $b$-tagged jets. Several orthogonal \glspl{sr} are optimized and statistically combined in the fit. 
Two discovery \glspl{sr} are also defined to provide stronger model-independent limits. 
If we assume that the higgsino decays to Higgs boson and gravitino with 100\% \gls{br}, 
this analysis excludes \mhino in the range 240-880 GeV at 95\% \gls{cl}. 
This analysis is complemented by a second analysis targeting signals with low \mhino, where the 
invisible momentum in the final state is not enough to fire the \met trigger. 
Because of a mild excess in this latter analysis, the excluded range of higgsino masses is 
between 130 and 230 GeV and between 290 and 880 GeV.
The results are also interpreted in models with a variable \gls{br} of the higgsino into Higgs or $Z$ boson. 
For \mhino = 400 GeV, signal models with \gls{br} to Higgs boson higher than 45\% are 
excluded at 95\% \gls{cl}.
These results are published in Ref. \cite{Aaboud:2018htj}. 

The results presented in this thesis are an important element in the wide \gls{atlas} program 
for \gls{susy} searches: they  
provide some of the most restrictive bounds on Natural \gls{susy} scenarios, 
and the experience gained in developing them represents a stepping stone to more sensitive searches 
with the data to be collected in the coming years.
So far no significant deviation from the \gls{sm} predictions has been found,
but the \gls{lhc} is still taking data, and the increase in luminosity
as well the effort in constantly improving the analysis techniques 
will allow to improve the sensitivity and probe also models that could have 
escaped detection in previous analyses. 
It should also always be kept in mind that all the model-dependent limits we present are 
based on simplified models with strong assumptions in terms of reachable particles and 
possible decay chains, and they typically become much weaker once we recast the existing analysis 
to more realistic models. 
The \gls{lhc} Run 2 will continue its $pp$-collision program until autumn 2018, 
providing an expected integrated luminosity of 140 \ifb; 
after this it will undergo a two-year-long shutdown, to resume operations 
in 2021 for three years of data taking at \cmfour TeV, with 2.5 times the 
nominal luminosity. 
During this period the size of the dataset collected by \gls{atlas} is expected to reach a total of approximately 300 \ifb, providing an unprecedented opportunity to explore the energy frontier and possibly leading to fascinating breakthroughs in our understanding of physics beyond the \gls{sm}. 
 

 




