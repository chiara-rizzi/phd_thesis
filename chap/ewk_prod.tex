\chapter{Electroweak-production multi-b SUSY search}
\label{chap:ewk_prod}


\section{Signal Model}

\begin{figure}[htbp]
	\centering
	\includegraphics[width=0.3\textwidth]{figures/ewk_prod/varie/N1N1-hhGG-bbbb_Z}
	\caption{Diagram for the simplified model considered in the analysis. The primary interpretation of the analysis is the decay via Higgs bosons, but decays via varied branching ratios to $Z$ bosons are also studied. The production of the \hino\ occurs
via mass-degenerate pairs of charginos or neutralinos, which decay to the \ninoone\ and immeasurably low momentum particles.} 
	\label{fig:feyn}
\end{figure}

\section{Previous Limits}

\section{Higgs Reconstruction}

\section{Analysis Variables}


\section{Pre-fit Data-MC}


\section{Signal Regions}

\begin{figure}[htbp]
	\centering
	\includegraphics[width=0.490\textwidth]{figures/ewk_prod/varie/schema-1}
	\caption{The division of signal, control, and validation regions using the $m(h_1)$ and $m(h_2)$ variables in the high-mass analysis.}
	\label{fig:binning_crvr}
\end{figure}

\begin{table}[h]
\caption{Signal region definitions for the high-mass analysis. The units of \met, \mtb, $m(h_1)$, $m(h_2)$, and \meffb are GeV. These variables are defined in Section~\ref{high_event_selection}.}
\begin{center}
\resizebox{1.\textwidth}{!}{
\begin{tabular}{|l|c|c|c|c|c|c|c|c|}
\hline
  & SR-3b-meff1-A & SR-3b-meff2-A & SR-3b-meff3-A & SR-4b-meff1-A & SR-4b-meff1-B & SR-4b-meff2-A & SR-4b-meff2-B  & SR-4b-meff1-A-disc \\
 \hline
\nbjet &  $=$3 &  $=$3 &  $\geq$3 &  $\geq$4 &  $\geq$4 &  $\geq$4 &  $\geq$4  & $\geq4$\\
 \hline
\met & \multicolumn{8}{|c|}{$>$ 200}\\
\hline
\dphimin    & \multicolumn{8}{|c|}{$>$0.4}\\
 \hline
\njet &  4--5 &  4--5 &  4--5 &  4--5 &  4--5 &  4--6 &  4--6 & 4--5\\
 \hline
\mtb &  $>$150 &  $>$150 &  $>$130 & - & - & - & - & - \\
 \hline
$m(h_1)$ &    \multicolumn{8}{|c|}{110--150}\\
 \hline
$m(h_2)$ &    \multicolumn{8}{|c|}{90--140}\\
 \hline
\dRmax &  0.4--1.4 &  0.4--1.4 &  0.4--1.4 &  0.4--1.4 &  1.4--2.4 &  0.4--1.4 &  1.4--2.4 & 0.4--1.4 \\
 \hline
\meffb &  600--850 &  850--1100 &  $>$1100 &  600--850 &  600--850 &  850--1100 &  850--1100 & $>600$ \\
 \hline
\end{tabular} 
}

\label{tab:SR}
\end{center}
\end{table}

\section{Control and validation regions}

\begin{table}[h]
\caption{Control region definitions in the high-mass analysis. The units of \met, \mtb, $m(h_1)$, $m(h_2)$, and \meffb are GeV. These variables are defined in Section~\ref{high_event_selection}.}
\begin{center}
\resizebox{0.75\textwidth}{!}{
\begin{tabular}{|l|c|c|c|c|c|}
\hline
  & CR-3b-meff1 & CR-3b-meff2 & CR-3b-meff3 & CR-4b-meff1 & CR-4b-meff2 \\
 \hline
\nbjet &  $=$3 &  $=$3 &  $\geq$3 &  $\geq$4 &  $\geq$4 \\
 \hline
\met  & \multicolumn{5}{|c|}{$>$ 200}\\
 \hline
\dphimin  & \multicolumn{5}{|c|}{$>$0.4}\\
 \hline
\njet &  4--5 &  4--5 &  4--5 &  4--5 &  4--6 \\
 \hline
\mtb &  $>$100 &  $>$100 &  $>$100 & - & - \\
 \hline
$m(h_1)$, $m(h_2)$  &  \multicolumn{5}{|c|}{ ($m(h_1)<$80, $m(h_2)<$80) or ($m(h_1)>$150, $m(h_2)<$80) or ($m(h_1)>$150, $m(h_2)>$140)    }\\
 \hline
\dRmax &  0.4--4 &  0.4--4 &  0.4--4 &  0.4--4 &  $\geq$ 0.4 \\
 \hline
\meffb &  600--850 &  850--1100 &  $>$1100 &  600--850 &  850--1100 \\
 \hline
\end{tabular} 
} 
\label{tab:CR}
\end{center}
\end{table}

\begin{table}[h]
\caption{Validation region definitions in the high-mass analysis. The units of \met, \mtb, $m(h_1)$, $m(h_2)$, and \meffb are GeV. These variables are defined in Section~\ref{high_event_selection}.}
\begin{center}
\resizebox{1\textwidth}{!}{
\begin{tabular}{|l|c|c|c|c|c|c|c|}
\hline
  & VR-3b-meff1-A & VR-3b-meff2-A & VR-3b-meff3-A & VR-4b-meff1-A & VR-4b-meff1-B & VR-4b-meff2-A & VR-4b-meff2-B \\
 \hline
\nbjet &  $=$3 &  $=$3 &  $\geq$3 &  $\geq$4 &  $\geq$4 &  $\geq$4 &  $\geq$4 \\
 \hline
\met  &  \multicolumn{7}{|c|}{$>$200}\\
 \hline
\dphimin &  \multicolumn{7}{|c|}{$>$0.4}\\
 \hline
\njet &  4--5 &  4--5 &  4--5 &  4--5 &  4--5 &  4--6 &  4--6 \\
 \hline
\mtb  & $>$120   & $>$100  & $>$80  & -  &  - & -  & -  \\
 \hline
$m(h_1)$, $m(h_2)$  &  \multicolumn{7}{|c|}{   (80<$m(h_1)$<150, $m(h_2)$<80) or ($m(h_1)$>150, 90<$m(h_2)$<140)   }\\
 \hline
\dRmax &  0.4--1.5 &  0.4--1.7 &  0.4--1.7 &  0.4--1.7 &  1.4--3 &  0.4--1.7 &  1.4--3 \\
 \hline
\meffb   & 550--900   & 800--1150  & $>$1050  & 550--900  & 550--900  & 800--1150  & 800--1150  \\
 \hline
\end{tabular} 
} 
\label{tab:VR}
\end{center}
\end{table}


\section{Results}

\begin{figure}[htbp]
	\centering
	\includegraphics[width=0.9\textwidth]{figures/ewk_prod/etmiss_results/histpull_pulls_in_CR_qcdStrong}
	\caption{Event yields in control regions and related \ttbar\
          normalization factors after the background-only fit for
          %Inputs and results of the likelihood fit in the control
          %regions of
          the high-mass analysis. The upper panel shows 
		the observed number of events and the predicted background yield before the fit.
		All uncertainties described in Section \ref{sec:syst_high} are included in the uncertainty band. The background category $\ttbar+X$ includes $\ttbar W/Z$, $\ttbar H$, and $\ttbar \ttbar$ events.  
		The $\ttbar$ normalization is obtained from the fit
                and is displayed in the bottom panel.
	} 
	\label{fig:pullCR}
\end{figure}


\begin{figure}[htbp]
	\centering
	\includegraphics[width=0.9\textwidth]{figures/ewk_prod/etmiss_results/histpull_pulls_in_VR_qcdStrong}
	\caption{Results of the background-only fit extrapolated to the VRs of the high-mass analysis. The $\ttbar$ normalization 
		is obtained from the fit to the CRs shown in Figure~\ref{fig:pullCR}. The upper panel shows 
		the observed number of events and the predicted background yield. The bottom panel shows the significance of any disagreement between the data and the background model~\cite{Choudalakis2012}.
		All uncertainties  defined in Section~\ref{sec:syst_high} are included in the 
		uncertainty band. The background category $\ttbar+X$ includes $\ttbar W/Z$, 
		$\ttbar H$, and $\ttbar \ttbar$ events. }
	\label{fig:pullVR}
\end{figure}

\begin{figure}[htbp]
	\centering
	% \subfigure[]{\includegraphics[width=0.65\textwidth]{figures/etmiss_results/histpull_pulls_in_CR_qcdStrong}\label{fig:pullCR}}\\
	\includegraphics[width=0.9\textwidth]{figures/ewk_prod/etmiss_results/histpull_pulls_in_SR_qcdStrong}
	\caption{Results of the background only fit extrapolated to the SRs of the high-mass analysis. The $\ttbar$ normalization
                is obtained from the fit to the CRs shown in Figure~\ref{fig:pullCR}. The data in the  SRs are 
	not included in the fit.  The upper panel shows the observed number of events and the predicted background 
	yield.  The bottom panel shows the significance of any disagreement between the data and the background model~\cite{Choudalakis2012}. All uncertainties  defined in Section~\ref{sec:syst_high} are included in the uncertainty band. The background
	category $\ttbar+X$ includes $\ttbar W/Z$, $\ttbar H$, and $\ttbar \ttbar$ events.} 
	\label{fig:pullSR}
\end{figure}

\begin{table}
\caption{Results of the background-only fit extrapolated to the SRs of the high-mass analysis, for the total background prediction and breakdown of the main background sources. 
	The uncertainties shown include all systematic uncertainties. The data in the SRs are not included in the fit. 
	The background category $\ttbar+X$ includes $\ttbar W/Z$, $\ttbar H$, and $\ttbar \ttbar$ events.
	The row ``MC-only background'' provides the total background prediction when the
	$\ttbar$ normalization is obtained from a theoretical
	calculation~\cite{Czakon:2011xx}.}
\label{tab:yieldsSR}
\resizebox{1.\textwidth}{!}{
\begin{tabular}{|l|c|c|c|c|c|c|c|c|}
\hline
SR name & SR-3b-meff1-A & SR-3b-meff2-A & SR-3b-meff3-A & SR-4b-meff1-A & SR-4b-meff1-B & SR-4b-meff2-A & SR-4b-meff2-B & SR-4b-meff1-A-disc\\
\hline
$N_{\mathrm{obs}}$ & 4 & 3 & 0 & 1 & 2 & 1 & 0 & 2\\
\hline
Total background & 2.6 $\pm$ 1.0 & 2.0 $\pm$ 0.5 & 0.8 $\pm$ 0.5 & 0.5 $\pm$ 0.4 & 3.2 $\pm$ 1.5 & 0.7 $\pm$ 0.5 & 2.0 $\pm$ 1.1 & 0.8 $\pm$ 0.7\\
Fitted \ttbar & 1.4 $\pm$ 0.8 & 0.89 $\pm$ 0.32 & 0.5 $\pm$ 0.4 & 0.35 $\pm$ 0.33 & 2.8 $\pm$ 1.5 & 0.6 $\pm$ 0.5 & 1.6 $\pm$ 1.0 & 0.6 $\pm$ 0.6\\
Single top & 0.43 $\pm$ 0.29 & 0.17 $\pm$ 0.14 & 0.040 $\pm$ 0.017 & $<$ 0.01 & 0.06 $\pm$ 0.13 & 0.030 $\pm$ 0.019 & $<$ 0.01 & 0.030 $\pm$ 0.019\\
$\ttbar+X$ & 0.39 $\pm$ 0.16 & 0.34 $\pm$ 0.14 & 0.09 $\pm$ 0.04 & 0.08 $\pm$ 0.06 & 0.24 $\pm$ 0.10 & 0.045 $\pm$ 0.025 & 0.039 $\pm$ 0.033 & 0.09 $\pm$ 0.06\\
$Z$+jets & 0.18 $\pm$ 0.14 & 0.21 $\pm$ 0.16 & 0.07 $\pm$ 0.20 & $<$ 0.01 & 0.09 $\pm$ 0.04 & $<$ 0.01 & $<$ 0.01 & 0.004 $\pm$ 0.011\\
$W$+jets & 0.20 $\pm$ 0.06 & 0.21 $\pm$ 0.09 & 0.08 $\pm$ 0.06 & 0.013 $\pm$ 0.009 & $<$ 0.01 & 0.022 $\pm$ 0.027 & 0.18 $\pm$ 0.10 & 0.013 $\pm$ 0.008\\
Diboson & $<$ 0.01 & 0.16 $\pm$ 0.11 & $<$ 0.01 & $<$ 0.01 & $<$ 0.01 & $<$ 0.01 & 0.17 $\pm$ 0.08 & $<$ 0.01\\
Multijet & $<$ 0.01 & 0.004 $\pm$ 0.005 & 0.004 $\pm$ 0.006 & 0.06 $\pm$ 0.05 & 0.0027 $\pm$ 0.0021 & 0.03 $\pm$ 0.04 & 0.007 $\pm$ 0.012 & 0.07 $\pm$ 0.05\\
\hline
MC-only background & 2.5 $\pm$ 1.0 & 2.0 $\pm$ 0.5 & 0.6 $\pm$ 0.4 & 0.43 $\pm$ 0.31 & 2.6 $\pm$ 0.9 & 0.43 $\pm$ 0.27 & 1.3 $\pm$ 0.6 & 0.7 $\pm$ 0.5\\
\hline
\end{tabular}
} 
\end{table}


\section{Systematic Uncertainties}

\begin{figure}[htbp]
	\centering
	\includegraphics[width=0.65\textwidth]{figures/ewk_prod/etmiss_misc/High-MET-syst.pdf}
	\caption{Relative systematic uncertainties in the background estimate for the high-mass analysis. The individual uncertainties can be correlated, such that the total background uncertainty is not necessarily their sum in quadrature. 
	} 
	\label{fig:syst_etmiss}
\end{figure}

\clearpage

\section{Interpretation}

\begin{table}
\begin{center}
\caption[Model independent upper limits]{For each discovery region, the number of observed events ($N_\mathrm{obs}$), the number of predicted events ($N_\mathrm{pred}$), and 95\% CL upper limits on the visible cross-section ($\sigma^\mathrm{95}_\mathrm{vis}$) and on the number of signal events ($S_\mathrm{obs}^\mathrm{95}$ ) are shown.  The fifth column ($S_\mathrm{exp}^\mathrm{95}$) shows the 95\% CL upper limit on the number of signal events given the expected number (and $\pm 1\sigma$ excursions of the expectation) of background events. The last column indicates the discovery $p$-value ($p(s~=~0)$) in significance units. The $p$-values are capped at 0.5. Results are obtained with $20\,000$ pseudoexperiments.}
\label{tab:UL_toys}
\begin{tabular}{
      lr
      S[table-format=4.1(1)]
      S[table-format=1.1(2)]
      S[table-format=2.1(1)]
      cc
      }
\toprule
{ Signal channel}           &   $N_\mathrm{obs}$ & \multicolumn{1}{c}{$N_\mathrm{pred}$}       & \multicolumn{1}{c}{$\sigma^\mathrm{95}_\mathrm{vis}$ [fb]}  &  $S_\mathrm{obs}^\mathrm{95}$  & $S_\mathrm{exp}^\mathrm{95}$ & $p_0$ (Z)  \\
\midrule
high-SR-4b-meff1-A-disc   &    2 &     0.8 \pm 0.7  & 0.15 &   5.5 & ${ 4.2 }^{ +1.3 }_{ -0.4 }$  &  0.15$~$(1.02) \\%
high-SR-3b-meff3-A        &    0 &     0.8 \pm 0.5  & 0.08 &   3.0 & ${ 3.1 }^{ +1.2 }_{ -0.1 }$  &  0.50$~$(0.00) \\%
low-SR-MET0-meff440       & 1063 &    1100 \pm 25   & 2.3  &  56   & ${ 79 }^{ +31 }_{ -23 }$     &  0.50$~$(0.00) \\%
low-SR-MET150-meff440     &   17 &      12 \pm 8    & 0.90 &  22   & ${ 19 }^{ +5 }_{ -4 }$       &  0.21$~$(0.80) \\%
\bottomrule
\end{tabular}
\end{center}
\end{table}


\begin{figure}[htbp]
	\centering
	\includegraphics[width=0.8\textwidth]{figures/ewk_prod/interpretation/limit_HM}
	\caption{The observed (solid black) vs expected (dashed black) 95\% upper limits on the total pair production cross section for degenerate higgsinos as a function of \mhino\ for the high-mass search. The 1 and 2$\sigma$ uncertainty bands are shown as green and yellow, respectively. Only the high-mass analysis results are used in this figure. The theory cross section is shown in the red curve. The bottom panel shows the ratio of the observed and expected limits with the theory cross section.} 
	\label{fig:exclusion_high}
\end{figure}

\begin{figure}[htbp]
	\centering
	\includegraphics[width=0.8\textwidth]{figures/ewk_prod/interpretation/br_limit_HM.pdf}
	\caption{BR limit} 
	\label{fig:exclusion_high}
\end{figure}



\section{Complementary Analysis Targeting Low Signal Masses}


\section{Combined Results}

\begin{figure}[htbp]
	\centering
	\subfigure[]{\includegraphics[width=0.49\textwidth]{figures/ewk_prod/interpretation/GGMupperLimit_unblinded_jump}\label{fig:exclusion_combined}}
	\subfigure[]{\includegraphics[width=0.49\textwidth]{figures/ewk_prod/interpretation/my_br_plot_unblind_yellow_band}\label{fig:exclusion_br}}
	\caption{Exclusion limits on \hino\ pair production. In both interpretations, the results of the low-mass analysis are used below $\mhino = 300$ GeV, while those of the high-mass analysis are used above. The figure shows (a) the observed (solid) vs expected (dashed) 95\% upper limits on the \hino\ pair production cross-section as a function of \mhino.  The 1$\sigma$ and 2$\sigma$ uncertainty bands on the expected limit are shown as green and yellow, respectively. The theory cross-section and its uncertainty are shown in the solid and shaded red curve. The bottom panel shows the ratio of the observed and expected limits with the theory cross-section. The figure also shows (b) the observed (solid) vs expected (dashed) 95\% limits in the \mhino\ vs $B(\hino\rightarrow h \tilde{G})$ plane, where $B(\hino\rightarrow h \tilde{G})$ denotes the branching ratio for the decay $\hino \rightarrow h \gravino$. The 1$\sigma$ uncertainty band is overlaid in green and the 2$\sigma$ in yellow. The regions above the lines are excluded by the analyses.} 
	\label{fig:exclusion}
\end{figure}

\section{Results in the Context of the ATLAS SUSY Group}