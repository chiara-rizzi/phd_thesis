\chapter{Statistics Tool Box}
\label{chap:stat}

The data we collect is stochastic: quantum mechanics is not deterministic, so the particle content result of an interaction follows probabilistic laws. Furthermore, experimental effects need to be taken into account. Therefore, a proper statistical treatment is essential to extract quantitative statements from the observed data. This section describes the main statistical procedures that are used to obtain the results described in Chapter \ref{chap:strong_prod} and Chapter \ref{chap:ewk_prod}. The two main topics discussed are parameter estimation (Section \ref{sec:stat:pe}), that has the aim to determine the value of the input parameters that allows to best describe data, and hypothesis testing (Section \ref{sec:stat:ht}), that checks the plausibility of models against the observed data.

\section{Parameter Estimation}

\label{sec:stat:pe}



\subsection{Inclusion of Systematic Uncertainties}

\subsection{Profiled Likelihood Ratio}

\section{Hypothesis Testing}
\label{sec:stat:ht}


\subsection{The CLs Method}

\subsection{Asymptotic Approximation}


