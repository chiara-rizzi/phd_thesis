\chapter{Event Reconstruction}
\label{sec:event:reco}

The particles produced in the \acrshort{pp} collisions in the center of the \acrshort{atlas} detector, interact with the detector material as discussed in Chapter \ref{chap:cern}. As a result of these interactions, electrical currents are recorded. \textit{Event reconstruction} is the process of recombining these digital signals and interpreting them as tracks and energy deposits in the calorimeters. A further step consists in analyzing the characteristics of the candidate tracks and calorimeter clusters and identify them as signs of the passage of specific particles.
This chapter describes the reconstruction and identification of the objects used in the analyses discussed in this thesis: tracks and vertices, electrons, muons, hadronic jets and missing transverse momentum. 



\subsection{Tracks and Vertices}
\label{sec:reco:tracks}

\subsection{Hadronic Jets}

\subsubsection{Jets from B-hadrons}


\subsection{Electrons}

\subsection{Muons}

\subsection{Missing Transverse Momentum}

Particles that interact only weakly with the detector, such as neutrinos or BSM particles like neutralinos, are not reconstructed directly. 
Their presence is instead inferred by measuring the total momentum imbalance in the event. 