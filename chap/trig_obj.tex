\chapter{Event Reconstruction}
\label{sec:event:reco}

The particles produced in the \gls{pp}
The analyses described in this thesis make use of 

Given a set of numbers, there are elementary methods to compute 
its \acrlong{gcd}, which is abbreviated \acrshortpl{gcd}. This 
process is similar to that used for the \acrfull{gcd}.


\subsection{Tracks and Vertices}
\label{sec:reco:tracks}

\subsection{Jets}

\subsubsection{B-tagging of Jets}

\subsubsection{Hadronic Taus}

\subsection{Electrons and Photons}

\subsection{Muons}

\subsection{Missing Transverse Momentum}

Particles that interact only weakly with the detector, such as neutrinos or BSM particles like neutralinos, are not reconstructed directly. Their presence is instead inferred by measuring the total momentum imbalance in the event. 